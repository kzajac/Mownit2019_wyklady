\section{Zmniejszenie korelacji w sekwencji liczb losowych}
	\begin{frame}{Zmniejszenie korelacji w sekwencji liczb losowych}
	- procedura ''{\it losowego tasowania}'' (random shuffling procedure) {\it Bays-Durham 	$\Rightarrow Knuth$: ``The Art of Computer 	Programing'' vol. II.} 
	\newline 

	\textbf{RANF} - generator systemowy,

	\textbf{RANO} - generator ulepszony

	\textbf{A} - tablica pomocnicza o dlugosci wyznaczonej przez liczbę pierwszą)

\newline \newline
	Uogólnienie:

	- kilka generatorów

	- jeden z nich wybiera ``dostarczyciela'' liczb
	\end{frame}
    
	\begin{frame}{Zmniejszenie korelacji w sekwencji liczb losowych}
		\centering \includegraphics[width=.8\linewidth]{img/14/14_5_1_img.png}
	\end{frame}

	\begin{frame}{Zmniejszenie korelacji w sekwencji liczb losowych}
		\begin{enumerate}
			\setcounter{enumi}{-1}
			\item wypełniam tablicę $A$ i zmienną $x$ liczbami losowymi
			\item $x$ traktuje jak indeks $j$, który wskazuje na  element tablicy A 
			\item $A(j)$ wstawiam w miejsce $x$ oraz jednocześnie zwracam jako liczbę losową  ulepszonego generatora
			\item z generatora systemowego RANF losujemy  brakującą  liczbę w miejsce $A(j)$
		\end{enumerate}
	\end{frame}

	
