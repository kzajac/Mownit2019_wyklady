\section{Zmienne losowe o zadanym rozkładzie}
\subsection{Metoda  tranformacji}
\begin{frame}{Rozkład równomierny na (0,1) - dystrybuanta}
    Dystrybuanta jednoznacznie definiuje rozkład prawdopodobieństwa:
$$
F(x)=\int_{-\infty}^{x}{p(y)dy}
$$
Funkcja niemalejąca, określa  $P(X<x)$\\
Dla rozkładu równomiernego na (0,1), dla
$x\in (0,1)$ $p(x)=1$
$$
F(x)=\int_{-\infty}^{x}{p(y)dy}=x
$$
Czyli dla $x \in (0,1)$
$$P(X<x)=x$$
\end{frame}
\begin{frame}{Metoda odwróconej dystrybuanty}


Jak uzyskać rozkład o zadanej dystrubuancie $F(x)$ ?

Jeśli zdefiniujemy U - zmienną losową o rozkładzie równomiernym na (0,1) to zmienna losowa
$$
X=F^{-1}(U)
$$
ma rozkład o dystrybuancie $F(x)$\\

Dowód:
$$
P(X<x)= P(F^{-1}(U)<x
)=P(U<F(x))=F(x)
$$

\end{frame}
\begin{frame}{Przykład rozkład wykładniczy $e^{-x}$}
Funkcja gęstości prawdopodobieństwa:    
    $p(x)=e^{-x}$ $x \in [0,\infty)$\\
    
Dystrybuanta:
$$
F(x)=\int_{-\infty}^x{e^{-x'}dx'}=1-e^{-x}
$$
$$
y=1-e^{-x}
$$
$$
x=-ln(1-y)
$$
$$
F^{-1}(y)=-ln(1-y)
$$
Generujemy ciąg liczb losowych o rozkładzie równomiernym $U$ $u_1, u_2, u_3, .., u_n \in (0,1)$\\
Wtedy ciąg liczb:
$y_i=-ln(1-u_i)$
ma rozkład wykładniczy.\\
W ogólności odwracanie dystrybuanty sprawia często duże trudności numeryczne. 
\end{frame}
\begin{frame}{Rozkład normalny}
    $$p(x) = e^{-\frac{x^{2}}{2}} \quad, \qquad$$
    Szukanie odwrotności dystrybuanty(funkcja nieelementarna!):
    $$F(x)=\int_{-\infty}^{x}e^{\frac{-y^2}{2}}dy'=erf(x) $$
    jest kosztowne.
    
    Zwykle stosuje się metodę Boxa-Mullera 
\end{frame}

\begin{frame}{Metoda Box-Muller}

		Biorę dwa niezależne rozkłady normalne i liczę prawdopodobieńswo 
		łączne (iloczynu zdarzeń):
		
		$$
			p(x_1,x_2) = e^{-\frac{x_1^{2}}{2}}\cdot e^{-\frac{x_2^{2}}{2}}=
			e^{-\frac{x_1^{2}+x_2^{2}}{2}}
		$$	
	    $$
		x_1, x_2 \in (-\infty,\infty)
		$$
		Wprowadzamy zmienne sferyczne:
		$$
		r^2=x_1^{2}+x_2^{2}, r \in[0,\infty)
		$$
		$$
		x_1=r\sin(\phi), \phi \in [0, 2\pi]
		$$
		$$
		x_2=r\cos(\phi)
		$$
\end{frame}
\begin{frame}{Metoda Box-Muller}
    Przeliczamy element prawdopodobieństwa (prawdopodobieństwo, że x i y  znajdą się w małym obszarze dxdy) we współrzędnych sferycznych (\colorbox{red}{$r$} - moduł Jakobianu)
$$
p(x,y)dxdy=p(r,\phi)
\textrm{\colorbox{red}{$r$}}
d\phi dr
$$

$$
e^{-\frac{r^2}{2}} \textrm{\colorbox{red}{$r$}} d\phi dr
$$
Wprowadzam zmienną $z=\frac{r^2}{2}$
$$
dz=rdr
$$
$$
e^{-z}  d\phi dz
$$
\end{frame}
\begin{frame}{Metoda Box-Muller }
    Otrzymaliśmy rozkład wykładniczy $e^{-z}  d\phi dz$.\\
    Możemy zastosować odwrotną dystrubuantę:
    $$
F^{-1}(w)=-ln(1-w)
$$
oraz odwrotną funkcję do:
$\frac{r^2}{2}=z$ czyli $r=\sqrt{2z} $\\
    Dodatkowo gęstośc prawdopodobieństwa rozkładu $ e^{\frac{r^2}{2}}$  nie zależy od $\phi \in [0,2\pi]$, który losujemy zgodnie z rozkładem równomiernym na przedziale $(0,2\pi)$
   % $$
   %F(x)= \int_{-\infty}^{x}\int_{0}^{2\pi}e^{-z} % d\phi dz=2\pi \int_{-\infty}^{x}e^{-z} dz
   % $$
\end{frame}
\begin{frame}{Metoda Box-Muller}
	\begin{block}{Metoda Box-Muller  generacji zmiennych losowych o rozkładzie normalnym}
		\[
			p(r)dr = e^{-\frac{r^{2}}{2}}dr= e^{-\frac{x_1^{2}+x_2^2}{2}}dx_1dx_2=
			e^{-\frac{x_1^{2}}{2}}dx_1 e^{-\frac{x_2^{2}}{2}}dx_2
		\]
	\end{block}

	\begin{block}{transformacja:}
		$U_{1}, U_{2}$ - zm. losowe niezależne, rozkł. równomierny na $(0, 1)$
		\[
			x_{1} = r cos(\phi)=\sqrt{-2 \ln(1-U_1)} \cos(2\pi U_2)
		\]
		\[
			x_{2} = r sin(\phi)= \sqrt{-2 \ln(1-U_1)} \sin(2\pi U_2)
		\]
			$\Rightarrow$ $x_{1} x_{2}$ - każda oddzielnie ma rozkład normalny (2 niezależne!)
	\end{block}
\end{frame}
%%%%%%%%%%%%%%
%\begin{comment}
%\begin{frame}{Metoda Box-Muller (1958)}
%	\begin{block}{stąd:}
%		\[
%			x_{1} = \left[- \frac{1}{2} (y_{1}^{2} + y_{2}^{2})\right]
%		\]
%		\[
	%		x_{2} = \frac{1}{2\pi} \arctan\frac{y_{1}}{y_{2}}
%		\]
%	\end{block}
	
%	\begin{block}{co daje:}
	%	\[
		%	\frac{\partial(x_{1}, x_{2})}{\partial(y_{1}, y_{2})} = \begin{vmatrix}
		%		\frac{\partial x_{1}}{\partial y_{1}} & \frac{\partial x_{1}}{\partial y_{2}} \\
		%		\frac{\partial x_{2}}{\partial y_{1}} & \frac{\partial x_{2}}{\partial y_{2}}
		%	\end{vmatrix} =
		%	- \left[\frac{1}{\sqrt{2\pi}} e^{-\frac{y_{1}^{2}}{2}}\right] \left[\frac{1}{\sqrt{2\pi}} e^{-\frac{y_{2}^{2}}{2}}\right]
	%	\]
	%	$\Rightarrow$ $y_{1} y_{2}$ - każda oddzielnie ma rozkład normalny (2 niezależne!)
%	\end{block}
%\end{frame}
%\end{comment}
%%%%%%%%%%%%%%
\begin{frame}{Uproszczenie obliczeń}
	\begin{block}{Zamiast:}
		$U_{1}, U_{2}$ - rozkład równomierny w jednostkowym kwadracie
	\end{block}

	\begin{block}{bierzemy:}
		$V_{1}, V_{2}$ - współrz. punktu w jednostkowym kole:\\
		$V_{1}^{2} + V_{2}^{2} < 1$,\\
		zamiast $U_1$ bierzemy $R = V_{1}^{2} + V_{2}^{2}$ (też ma rozkład równomierny)\\
		zamiast $U_{2}$ - $\angle(V_{1}, V_{2})$ 
		\[
			\cos(2\pi U_{2}) = \frac{V_{1}}{\sqrt{R}}; \qquad \sin(2\pi U_{2}) = \frac{V_{2}}{\sqrt{R}}
		\]
		i tak unikamy stosowania funkcji trygonometrycznych
	\end{block}
\end{frame}
%%%%%%%%%%%%%%
\begin{frame}{Uproszczenie obliczeń}
	\begin{block}{}
		\[
			x_{1} = V_{1}\sqrt{\frac{-2 \ln(V_{1}^{2} + V_{2}^{2})}{V_{1}^{2} + V_{2}^{2}}},
		\]
		\[
			x_{2} = x_{1} \cdot \frac{V_{2}}{V_{1}}
		\]
		$x_{1}, x_{2}$ - niezależne, obie o $N(0,1)$
	\end{block}
\end{frame}