\section{Zasady doboru stałych generatora liniowego kongruentnego}
	\begin{frame}{Zasady doboru stałych generatora liniowego kongruentnego}
Pojawiające sie w literaturze wnioski:\\ 
	\newline \newline
 	 $I_{0}$: \\
 	 - nie ma większego znaczenia\\
	
	$a$: \\
	- $a(mod 8)=5$ \\
	- $\frac{m}{100}<a<m-\sqrt{m}$ \\
	- brak powtarzającego się wzorca w zapisie w systemie dwójkowym
	\newline \newline
	$c$: \\
	- nieparzyste \\
 	- spełniające $ \frac{c}{m}\approx\frac{1}{2}-\frac{\sqrt{3}}{6}$
	\newline \newline
	$m$: \\
	$m=2^{t}$ , \quad t-liczba bitów przeznaczonych na 1 liczbę całkowitą


	\end{frame}
