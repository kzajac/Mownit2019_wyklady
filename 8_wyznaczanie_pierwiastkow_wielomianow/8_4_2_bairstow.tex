\section{Metoda Bairstowa oraz deflacja czynnikiem kwadratowym}

\begin{frame}{Metoda Bairstowa}
%  \textit{(Można użyć również metody N-R.)}

  \vspace{5px}

  \textbf{Metoda Bairstowa} polega na szukaniu czynników kwadratowych
Czynnik kwadratowy może objąć 2 pierwiastki rzeczywiste lub zespolone, np.
  $$x_1 = \alpha + \textit{\textrm{i}}\beta, \quad x_2 = \alpha - \textit{\textrm{i}}\beta$$
  $$(x - x_1)(x - x_2) = x^2 - 2 \cdot \alpha x + (\alpha^2 + \beta^2) = x^2 + p \cdot x + q$$
Dzięki "wyłuskaniu" czynnika kwadratowego o pierwiastkach zespolonych możemy jak najdłużej pozostać w arytmetyce rzeczywistej
  

  \begin{equation}
    P(x) = (x^2 + p \cdot x + q) \cdot Q(x) + R \cdot x + S \label{bairstow}
  \end{equation}
\end{frame}

\begin{frame}
  $$\left. \begin{array}{l}
  R(p,q)=0 \\ S(p,q)=0
  \end{array}\right\} \Rightarrow \text{met. Newtona-Raphsona dla dwoch zmiennych}$$
  
  Przypomnienie dla jednej zmiennej
  $$x_{i+1}=x_i-\frac{p(x_i)}{p'(x_i)}$$

  \vspace{5mm}
 Dla dwóch zmiennych:
  $$\left( \begin{array}{l}
  p_{(n+1)} \\ q_{(n+1)}
  \end{array} \right)
  =
  \left( \begin{array}{l}
  p_{(n)} \\ q_{(n)}
  \end{array} \right)
  -
  \left( \begin{array}{ll}
  \frac{{\partial}R}{{\partial}p} & \frac{{\partial}R}{{\partial}q} \\
  \frac{{\partial}S}{{\partial}p} & \frac{{\partial}S}{{\partial}q}
  \end{array} \right)^{-1}
  \left( \begin{array}{l}
  R(p_n,q_n) \\ S(p_n,q_n)
  \end{array} \right)$$
\end{frame}

\begin{frame}
 W każdym kroku $ R(p_n,q_n)$ oraz  $S(p_n,q_n)$ znajdujemy poprzez  syntetyczne dzielenie wielomianu $P(x)$ przez $(x^2 + p_n \cdot x + q_n)$\\
  
  Pochodne znajdujemy w następujący sposób:

  $P(x)$ -- nie zależy od $p$, $q$, \\ Z równania (\ref{bairstow}) mamy:

  $$\left. \begin{array}{l}
  0 = (x^2 + p \cdot x + q) \cdot \frac{{\partial}Q}{{\partial}q} + Q + x \cdot \frac{{\partial}R}{{\partial}q} + \frac{{\partial}S}{{\partial}q} \\
  0 = (x^2 + p \cdot x + q) \cdot \frac{{\partial}Q}{{\partial}p} + x \cdot Q + x \cdot \frac{{\partial}R}{{\partial}p} + \frac{{\partial}S}{{\partial}p}
  \end{array}\right\}
  \begin{array}{l}
    \left( \leftarrow \frac{{\partial}P}{{\partial}q} \right) \\
    \left( \leftarrow \frac{{\partial}P}{{\partial}p} \right)
  \end{array}$$

  \begin{equation}
    \left. \begin{array}{l}
    (x^2 + p \cdot x + q) \cdot \frac{{\partial}Q}{{\partial}q} + \frac{{\partial}R}{{\partial}q} \cdot x + \frac{{\partial}S}{{\partial}q} = -Q(x) \\
    (x^2 + p \cdot x + q) \cdot \frac{{\partial}Q}{{\partial}p} + \frac{{\partial}R}{{\partial}p} \cdot x + \frac{{\partial}S}{{\partial}p} = -x \cdot Q(x)
    \end{array}\right\}
    \label{bairstow2}
  \end{equation}
\end{frame}

\begin{frame}
  (\ref{bairstow2}) mają podobną strukturę jak (\ref{bairstow}) $\rightarrow$ pochodne $\frac{{\partial}R}{{\partial}p}$, $\frac{{\partial}R}{{\partial}q}$, $\frac{{\partial}S}{{\partial}p}$, $\frac{{\partial}S}{{\partial}q}$ \\ \vspace{2mm} mogą być uzyskane przez syntetyczne dzielenie wielomianów \\ \vspace{3mm} $\left.\begin{array}{l} -Q(x) \\ -x Q(x) \end{array} \right\}$ przez czynnik kwadratowy.
\end{frame}
