\section{Procedura Maehly'ego}

\begin{frame}{Pro. Maehly'ego -- technika wygładzania pierwiastków}
Wykorzystujemy znane pierwiastki bez stosowania deflacji.
  Równocześnie zapobiega zlewaniu się w jeden dwóch różnych pierwiastków (na~etapie wygładzania)\\
\vspace{4mm}
Zredukowany wielomian
    $$P_j(x) \equiv \frac{P(x)}{(x - x_1)\ldots(x - x_j)}$$
Pochodna
  $$P'_j(x) = \frac{P'(x)}{(x - x_1)\ldots(x - x_j)} - \frac{P(x)}{(x - x_1)\ldots(x - x_j)} \sum_{i=1}^j \frac{1}{x - x_i}$$
\end{frame}

\begin{frame} 
Wykorzystujemy znane pierwiastki $(x_1 ,..., x_j)$ do znalezienia kolejnych.\\
  Pojedynczy krok metody Newtona-Raphsona można zapisać używając zredukowanego wielomianu:

$$    x_{k+1} = x_k - \frac{P_j(x_k)}{P'_j(x_k)}$$

Zastępujemy zredukowany wielomian początkowym wielomianem, ale uwzględniamy informację o już znalezionych pierwiastkach:
\begin{gather*}
    x_{k+1} = x_k - \frac{P(x_k)}{P'(x_k) - P(x_k) \cdot \sum_{i=1}^{j}\frac{1}{(x_k - x_i)}}
  \end{gather*}

\end{frame}
