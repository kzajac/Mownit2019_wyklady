\subsection{Pierwiastki rzeczywiste -- metoda Newtona-Raphsona}

\begin{frame}{Wygładzanie pierwiastków (root polishing) \\Pierwiastki rzeczywiste -- metoda Newtona-Raphsona}
Metoda Newtona
    $$t_{i+1}=t_{i}-\frac{p(t_i)}{p'(t_i)}$$
    Wykonujemy schemat Hornera aby obliczyć wartość $p(t)=\sum_{i=0}^m a_i\cdot t^i$.
    $$ \left \{ \begin{array}{l}
    b_{m-1} = a_m \\
    b_i = a_{i+1} + t \cdot b_{i+1}, \quad i = m - 2, m-3, \dots, 0 \\
    p(t) = a_{0} + t \cdot b_{0}
    \end{array} \right. $$
    \end{frame}
    \begin{frame}
     Dla $t$ -- pochodną $p'(t)=q(t)$ można obliczyć przy wyliczaniu $p(t)$ również schematem Hornera, ponieważ od razu wyliczamy i korzystamy z $b_i$
      \begin{gather*}
      p(x)= (x - t) q(x) + p(t)\\
      p'(x) = q(x) + (x - t) q'(x)\\
      p'(t) = q(t)=\sum_{i=0}^{m-1} b_i\cdot t^i\\
    \end{gather*} 
    $$ \left \{ \begin{array}{l}
    c_{m-2} = b_{m-1} \\
    c_i = b_{i+1} + t \cdot c_{i+1}, \quad i = m - 3, m-4, \dots, 0 \\
    q(t) = b_{0} + t \cdot c_{0}
    \end{array} \right. $$
\end{frame}

\begin{frame}[fragile]
  Pojedynczy krok algorytmu:
  \begin{lstlisting}
    //Horner (uwaga: przenumerowanie indeksow !)
    b[n] := a[n]
    c[n] := b[n]
    for k = n - 1 downto 1
      b[k] := a[k] + t * b[k + 1]
      c[k] := b[k] + t * c[k + 1]
    b[0] := a[0] + t * b[1]

    //b[0] == p(t),
    //c[1] == p'(t)
    p := b[0]
    p1 := c[1]

    if p1 == 0
      raise error("Derivative should not vanish.")

    //Newton-Raphson step
    t := t - p / p1
  \end{lstlisting}

  \textbf{Zadanie:} Sprawdzić poprawność.
\end{frame}
