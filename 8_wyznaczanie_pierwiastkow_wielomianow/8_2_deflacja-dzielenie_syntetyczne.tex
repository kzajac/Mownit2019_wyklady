\section{Deflacja -- dzielenie syntetyczne}
\begin{frame}{Przydatność deflacji}
\textbf{Deflacja} - obniżenie stopnia wielomianu po znalezieniu jego pierwiastka; bardzo użyteczna część algorytmu wyznaczania pierwiastków wielomianów. \\

$P(x)$ -- wielomian,\\
  $r$ -- znaleziony pierwiastek wielomianu $P$.

  Realizujemy \textit{faktoryzację}:
  \begin{block}{}
    $$ P(x) = (x - r) \cdot Q(x) $$
  \end{block}

  \begin{itemize}
    \item Zmniejszenie złożoności obliczeniowej\\
     ($Q$ -- niższego stopnia niż $P$)
    \item Uniknięcie pomyłki -- powrotu do pierwiastka już znalezionego.
  \end{itemize}
\end{frame}
\begin{frame}{Algorytm Hornera - przypomnienie}
  \begin{block}
    $$ f(x) = \sum_{i=0}^m a_i \cdot x^i = ( ( \dots (( a_m \cdot x + a_{m-1}) \cdot x + a_{m-2} ) \cdot \dots) \cdot x + a_1) \cdot x + a_0 $$
  \end{block}

  Rekurencyjny algorytm obliczania wartości wielomianu dla $x = \lambda$ ma postać:

  \begin{block}{Algorytm}
    $$ \left \{ \begin{array}{l}
    b_{m-1} = a_m \\
    b_i = a_{i+1} + \lambda \cdot b_{i+1}, \quad i = m - 2, m-3, \dots, 0 \\
    f( \lambda ) = a_{0} + \lambda \cdot b_{0}
    \end{array} \right. $$
  \end{block}
\end{frame}

\begin{frame}

  \begin{block}{Twierdzenie}
    $$ f(x) - f(\lambda) = (x - \lambda) \sum_{i=0}^{m-1} b_i x^i $$
  \end{block}

  Dowód (szkic):
{\scriptsize
  $$ f(x) = \sum_{i=0}^{m-1} b_i x^i (x - \lambda)+ f(\lambda)$$
  $$ f(x) = (b_{m-1}x^{m-1}+ b_{m-2}x^{m-2}+...+b_0)(x - \lambda)+ f(\lambda)$$
  $$ f(x) = (b_{m-1}x^{m}+ b_{m-2}x^{m-1}+...+b_0 x)- \lambda (b_{m-1}x^{m-1}+ b_{m-2}x^{m-2}+...+b_0)+ f(\lambda)$$
  $$f(x)= \underbrace{b_{m-1}}_{a_m}x^m+\underbrace{(b_{m-2}-\lambda b_{m-1})}_{a_{m-1}}x^{m-1}+...-\lambda b_0+f(\lambda)$$
  }
  Czyli:
$$ b_{i-1} = a_i + \lambda \cdot b_i $$
 Wniosek: można stosować algorytm Hornera do dzielenia wielomianu przez czynnik liniowy (x-\lambda)$
\end{frame}
