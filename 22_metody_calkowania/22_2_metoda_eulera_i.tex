\section{Metoda Eulera pierwszego rzędu}
%%%%%%%%%%%%%%%%%%%%%%%%%%
\begin{frame}{Metoda Eulera pierwszego rzędu}
  \underline{$f(u,t)$= ?} \quad dla  \quad $t \in [t^n,t^{n+1}] \approx f(u^n,t^n)$
  \begin{figure}
	\includegraphics[height=0.65\textheight]{img/22/img1.jpg}
	\end{figure}
\end{frame}
%%%%%%%%%%%%%%%%%%%%%%%%%%
\begin{frame}{Cechy metody}
  \textbf{Algorytm:}
  $$u^{n+1} = u^n - f(u^n, t^n) \cdot \Delta t$$
  \textbf{Cechy:}
  \begin{itemize}
    \item jawna
    \item 1-go rzędu - bład zmienia liniowo się ze względu na $\Delta t$: \quad $\varepsilon=0(\Delta t)$
    \item prosta
    \item efektywna
  \end{itemize}
\end{frame}
%%%%%%%%%%%%%%%%%%%%%%%%%%
\begin{frame}{Stabilność}
  w czasie $t^n$ mamy  $u^n$ z błędem $\varepsilon^n$  \\
  w czasie $t^{n+1}$ mamy $u^{n+1}$ z błędem $\varepsilon^{n+1}$ \\
  \fbox{$\varepsilon^{n+1} = g \cdot \varepsilon^n$\qquad
  g - współczynnik wzmocnienia błędu}
  $$u^{n+1} + \varepsilon^{n+1} = u^n+\varepsilon^n-f(u^n+\varepsilon^n,t^n)\cdot \Delta t \qquad(*)$$
  \underline{zał.:} $\varepsilon^n$ - mały $\Rightarrow$ rozwijamy w szereg Taylora, pomijamy człony nieliniowe:
  $$f(u^n+\varepsilon^n,t^n) = f(u^n,t^n)+\frac{\partial f}{\partial u}{\bigg\arrowvert}_{u^n} \cdot \varepsilon^n$$
  Po podst. do (*):
   $$\textrm{\colorbox{red}{$u^{n+1}$}} + \varepsilon^{n+1} = \textrm{\colorbox{red}{$u^{n}$}}+\varepsilon^n-(\textrm{\colorbox{red}{$f(u^n,t^n)$}}+\frac{\partial f}{\partial u}{\bigg\arrowvert}_{u^n} \cdot \varepsilon^n) \cdot \textrm{\colorbox{red}{$\Delta t$}}$$
  $$\varepsilon^{n+1} = \varepsilon^n-\frac{\partial f}{\partial u}\big\arrowvert_n \cdot \Delta t \cdot \varepsilon^n $$
  \end{frame}
  \begin{frame}
  \underline{współczynnik wzmocnienia}:\qquad $g = \frac{\epsilon^{n+1}}{\epsilon^{n}}=1- \frac{\partial f}{\partial u}{\big\arrowvert}_n \cdot \Delta t$\\
  warunek stabilności: $|g|\leq1$\\
  warunek stabilności dla $\frac{\partial f}{\partial u}>0$
  $$\frac{\partial f}{\partial u}\arrowvert_n \cdot \Delta t \leqslant 2 \quad 
  \rightarrow \quad \Delta t \leqslant \frac{2}{\frac{\partial f}{\partial u}\arrowvert_n} \Rightarrow krok$$
  (gdy $\frac{\partial f}{\partial u}\arrowvert_n < 0$ - metoda niestabilna)\\
   \underline{stabilność} - fundamentalna własność metody różnicowej
\end{frame}
%%%%%%%%%%%%%%%%%%%%%%%%%%

\begin{frame}{Przykład}
	\begin{center}
	$\frac{du}{dt}+\frac{u}{\tau} = 0$, \quad $u(0) = 1$
	%\newline $R$, $L\rightarrow \tau = %\frac{L}{R}$\qquad 
	\newline 
	rozpad promieniotwórczy\par
      \begin{itemize}
    \item rozwiązanie otrzymane analitycznie: $u = e^{-\frac{t}{\tau}}$
    \end{itemize}
	\end{center}
  
  krok czasowy gwarantujący stabilność:
  $$
  \Delta t \leqslant \frac{2}{\frac{\partial f}{\partial u}\arrowvert_n}$$
  $$\frac{\partial f}{\partial u}\bigg\arrowvert_n = \frac{1}{\tau}\qquad \Delta t \leqslant 2\tau$$
 
\end{frame}
%%%%%%%%%%%%%%%%%%%%%%%%%%

\begin{frame}{Przykład 2 - równanie typu oscylacyjnego}
	\begin{center}
		oscylator harmoniczny: \qquad $\frac{d^2x}{dt^2}+ \omega^2x=0$
	\end{center}
    $\Rightarrow$ układ 2 równań (dla $v=\frac{dx}{dt}$):
    $$\left\{\begin{array}{lcclll}
	\frac{dv}{dt}&+&\omega^2x&=&0 &\\
	\frac{dx}{dt}&-&v&=&0 & \mid \cdot (-i \omega )\\
	\end{array} \right.$$
	po dodaniu mamy:
	$$ 	\frac{dv}{dt}-i \omega \frac{dx}{dt}+
	\omega^2 x+i \omega v  = 0$$
	$$ 	\frac{dv}{dt}-i \omega \frac{dx}{dt}+
	-i^2 \omega^2 x+i \omega v  = 0$$
	$$ 	\frac{dv}{dt}-i \omega \frac{dx}{dt}+
	i\omega( v-i\omega x ) = 0$$
	 \end{frame}
	 %%%%%%%%%%%%%%%%%%%%%%%%%%
    \begin{frame}{Przykład 2 c.d.}
    wprowadzenie \underline{$u = v-i\omega x$} $\Rightarrow$ pojedyncze równanie 1-go rzędu:
    $$ \frac{du}{dt} + i\omega u= 0 \qquad  \qquad $$
\underline{współczynnik wzmocnienia:} ($f(u)=i\omega u$)
    $$g = 1-{\frac{\partial f}{\partial u}}\bigg\arrowvert _n \cdot \Delta t \quad \Rightarrow \quad g=1-i\cdot \omega \cdot \Delta t \qquad zespolony!$$
	$$|g|^2 = g \cdot g^* = 1+\omega^2 {\Delta t}^2 \quad 
 \Rightarrow |g| > 1 $$
  \fbox{Metoda Eulera jest  niestabilna dla równań typu oscylacyjnego.}\\
	%\rightarrow 
	%\quad 1+\bigg(\frac{\partial f}{\partial %u}\bigg{\arrowvert}_n \cdot \Delta t\bigg)^2 > 1
    W zagadnieniach \textit{nieliniowych $\frac{\partial f}{\partial u}$} jest funkcją $u$ \quad $\Rightarrow$ \quad należy na każdym etapie wybierać $\Delta t$ spełniające warunki stabilności. 
\end{frame}
%%%%%%%%%%%%%%%%%%%%%%%%%%
