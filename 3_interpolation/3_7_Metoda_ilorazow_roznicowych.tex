\section{Metoda ilorazów różnicowych (divided differences, Newtona)}
\begin{frame}
{3.7 Metoda ilorazów różnicowych (divided differences, Newtona)}

Interesuje nas nie wartość, a  postać wielomianu (dobra do celów praktycznych)

$P_{n}(x)-\text{LIP, stp. } \leq n$, zgodny z $f(x)$ w $\{x_{0},\ x_{1},\ .\ .\ .\ ,\ x_{n}\}$

Można go zapisać w postaci:
\begin{equation*}\begin{split}
P_{n}(x)=a_{0}+a_{1}(x-x_{0})+a_{2}(x-x_{0})(x-x_{1})+ \\
\cdots+a_{n}(x-x_{0})(x-x_{1})\cdots(x-x_{n-1})
\end{split}
\end{equation*}

$a_{0}$: $f(x_{0})=P_{n}(x_{0})=a_{0}$

$a_{1}$ : $f(x_{1})=P_{n}(x_{1})=a_{0}+a_{1}(x_{1}-x_{0})=f(x_{0})+a_{1}(x_{1}-x_{0}) \Rightarrow$

$$
a_{1}=\frac{f(x_{1})-f(x_{0})}{x_{1}-x_{0}}
$$
\end{frame}

\begin{frame}
Wprowadzamy notację:
\begin{itemize}
\item 0-wy iloraz różnicowy wzgl. $x_{i}$ : $f[x_{i}]=f(x_{i})$ \\
pozostałe - indukcyjnie:

\item 1-szy:
$$
f[x_{i},\ x_{i+1}]=\frac{f[x_{i+1}]-f[x_{i}]}{x_{i+1}-x_{i}}
$$
Gdy zaś określone są ilorazy aż do $(k-1)$ , czyli
$$
f[x_{i},\ x_{i+1},\ x_{i+k-1}] \: i \: f[x_{i+1},\ x_{i+2},\ x_{i+k}]
$$
\item to wtedy k-ty iloraz różnicowy:
\end{itemize}
$$
f[x_{i},\ x_{i+1}, ...\ x_{i+k}]=\frac{f[x_{i+1},x_{i+2},\ldots.x_{i+k}]-f[x_{i},x_{i+1},\ldots.x_{i+k-1}]}{x_{i+k}-x_{i}}
$$
\end{frame}
\begin{frame}{Budowa tablicy ilorazów różnicowych}
\begin{itemize}
    \item wystarczy zrobić tylko raz dla danego zestawu węzłów interpolacji
    \item do wzoru wykorzystujemy wyniki na przekątnej (czerwone)
    \item dodanie nowego węzła nie wymaga powtarzania obliczeń od początku
    
\end{itemize}

  \begin{array}{llllll} 
  x_0 & \color{red} f(x_0) \\ 
  x_1 & f(x_1) & \color{red}  f[x_0,x_1] \\ 
  x_2 & f(x_2) & f[x_1,x_2] & \color{red}  f[x_0,x_1,x_2] \\ 
  \ldots & \ldots & \ldots & \ldots & \ldots & \ldots\\ 
  x_n & f(x_n) & f[x_{n-1},x_n] & \ldots & \ldots & \color{red}  f[x_0,\ldots,x_n] 
  \end{array}
\end{frame}
\begin{frame}
Interpolacyjny wzór Newtona z ilorazami różnicowymi:
$$
 P_{n}(x)=
   f[x_{0}]+(x-x_0) f[x_{0}, x_{1}]+(x-x_0) (x-x_1)f[x_{0},x_{1},x_{2}]+ \\
+ \cdots +(x-x_{0})(x-x_{1})\cdots(x-x_{n-1})f[x_{0},x_{1},\dots ,x_{n}]
$$\\
$$
P_{n}(x)=f[x_{0}]+\sum_{k=1}^{n}f[x_{0},\ x_{1},\ .\ .\ .\ ,\ x_{k}](x-x_{0})\cdots(x-x_{k-1})
$$
Postać Newtona $\rightarrow$ do obliczania wartości wielomianu najlepiej użyć wariantu schematu Hornera
%\textbf{Zadanie}: Policzyć $a_{2}, a_{3}$, zapisać algorytm.

\end{frame}


\begin{frame}
Związek ilorazów różnicowych z pochodnymi.
\begin{block}
{Generalized Mean value theorem}

Założenia:
\begin{itemize}
\item $f\in C^{n}[a,\ b]$
\item $x_{0}, x_{1}$, . . . , $x_{n}\in[a,\ b] $ i są różne
\end{itemize}

Teza:
$$
\exists\eta\in(a,\ b)\ f[x_{0},\ x_{1},\ .\ .\ .\ ,\ x_{n}]=\frac{f^{(n)}(\eta)}{n!}
$$
\end{block}
\vspace{5mm}

\textbf{Materiały o potrzebnych twierdzeniach.} \url{https://tinyurl.com/tb95zoe}
\end{frame}

\begin{frame}
\textbf{Wzor Newtona dla węzłów równoodległych}

$h=x_{i+1}-x_{i} \quad i=0, 1, ..., n-1$ \\
$x_i=x_0+i\cdot h$\\
$x=x_{0}+s\cdot h$ \\
$x-x_{i}=(s-i)h$ \\

\begin{gather*} P_{n}(x)=P_{n}(x_{0}+s\cdot h) =\\
   f[x_{0}]+s\cdot h\cdot f[x_{0}, x_{1}]+s(s - 1)h^{2}f[x_{0},x_{1},x_{2}]+ \\
+ \cdots +s(s-1)\cdots(s-n+1)h^{n}f[x_{0},x_{1},\dots ,x_{n}]
\end{gather*}

\begin{gather*}
\binom{s}{k}=\displaystyle \frac{s(s-1)\ldots(s-k+1)}{k!}\\
P_{n}(x)=P_{n}(x_{0}+sh)=\sum_{k=0}^{n}(_{k}^{s})k!h^{k}f[x_{0}, x_{1}, \cdots ,x_{k}](**)
\end{gather*}
\end{frame}
\begin{frame}{Różnica progresywna(forward difference)}
Progresywna, bo pomiędzy $i$ oraz $i+1$ (do przodu):
  \begin{gather*}
    \Delta^{(0)} y_i := y_i\\
    \Delta^{(k)} y_i := \Delta^{(k - 1)} y_{i+1} - \Delta^{(k - 1)} y_i, k \ge 1
  \end{gather*}

  Forward differences -- przykład dla 4 punktów:
  \begin{figure}[h]
  			\includegraphics[width = 0.4 \linewidth]{img/3/interpol_3_7}
  	\end{figure}
\end{frame}
\begin{frame}{Newton forward divided-difference formula }
\begin{itemize}
\item Dla węzłów równoodległych iloraz różnicowy jest równy:
$$f[x_{0},\displaystyle \ x_{1},\cdots,\ x_{k}]=\frac{1}{k!h^{k}}\triangle^{k}f(x_{0})$$ \\

\item Podstawiając do (**) mamy:
$$P_{n}(x)=P_n(x_0+s\cdot h)=\displaystyle
\sum_{k=0}^{n}(_{k}^{s})\triangle^{k}f(x_{0})$$
\item Dysponując wartościami różnic progresywnych można wyznaczyć wartość wielomianu.
\end{itemize}
\end{frame}


