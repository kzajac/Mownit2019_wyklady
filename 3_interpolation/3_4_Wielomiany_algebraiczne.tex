\section{Wielomiany algebraiczne}
	\begin{frame}{3.4 Wielomiany algebraiczne \\ 3.4.1 Cechy wielomianów algebraicznych}


	\begin{itemize}
	\item łatwość obliczeń $+, *, \displaystyle \frac{d}{dx}, \displaystyle \int dx\ldots $ \newline


    \begin{block}{Twierdzenie Weierstrass'a}
    Dla dowolnej $f(x)$ -- ciągłej na $[a,\ b]$ (skończonym) i każdego

    $\epsilon>0$, istnieje wielomian $W_{n}$, $n=n(\epsilon)$ taki, że:

	\[ \max\limits_{x \in [a,b]}|f(x)-W_{n}(x)|<\epsilon \]


  	\end{block}
    \end{itemize}

	\end{frame}

    \begin{frame}{3.4.2 Postać naturalna wielomianu}
    \begin{block}{}
		$$W(x)=\sum_{i=0}^{n}a_{i}x^{i},\ a_{i}=\frac{W^{(i)}(0)}{i!}$$
    \end{block}
        \begin{itemize}
        \item postać naturalna - rozwinięcie Maclaurina \\

        \item $a_{i}$ - znormalizowane pochodne
		\end{itemize}
    \end{frame}

    \begin{frame}{3.4.3 Algorytm W.G. Hornera}
    Do wyliczania wartości wielomianu dla konkretnego x
    \setlength\parindent{24pt}
	\begin{block}{}
		$$W(x)=(\ldots(a_{n}x+a_{n-1})x+\cdots+a_{1})x+a_{0}$$
	\end{block}
	czyli:

	$\qquad W_{n}=a_{n},$ \\

	$\qquad W_{i}=W_{i+1}x+a_{i}, \quad i=n-1, n-2, \cdots , 0$ \\

	$\qquad W(x)=W_{0}$ \\

	otrzymujemy:
	\begin{itemize}
	\item $n-$mnożeń, $n -$dodawań
	\item numerycznie poprawny (wskaźnik kumulacji $\approx 2n+1$)
	\end{itemize}
    \vspace{3mm}
	$\quad$\textbf{Zadanie:} sprawdzić.
    \end{frame}

    \begin{frame}{3.4.4 Postać Newtona}
    \begin{block}{}
	$$W(x)\ =\ \sum_{k=0}^{n}b_{k}p_{k}(x)\ ;$$
	\end{block}
	$$p_{0}(x)\ =\ 1$$

	$$p_{k}(x)\ =\ (x-x_{0})(x-x_{1})\ldots(x-x_{k-1})$$

    \end{frame}
\begin{frame}{Algorytm Hornera dla postaci Newtona }
   $\qquad W_n = b_n$\\
	$\qquad W_i\, = \,W_{i+1}\cdot (x-x_i)\,+\,b_i$\\
	$\qquad W(x)=W_{0}$ \\
\end{frame}