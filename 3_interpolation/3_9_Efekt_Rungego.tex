\section{Efekt Rungego}
\begin{frame}
{3.9 Efekt Rungego}

\begin{block}{}
	\textbf{Intuicja: } Zwiększenie liczby węzłów $\Rightarrow$ lepsza dokładność. \\
	\textbf{Rzeczywistość: } Efekt Rungego -- zmniejszenie dokładności.
\end{block}

(C. Runge, 1901)
$$
y(x)=\frac{1}{1+25x^{2}},\ x\in[-1,\ 1]
$$
Występuje, gdy:
\begin{itemize}
\item interpolujemy wielomianami

\item węzły interpolacyjne są równoodległe
\end{itemize}
\end{frame}

\begin{frame}
	\begin{figure}[h]
		\includegraphics[scale=0.35]{img/3/interpol_3_9}
	\end{figure}
\end{frame}

\begin{frame}
\begin{itemize}
\item Katastrofalna rozbieżność dla $0.726\leq|x|<1$.

\item Lecz bardzo dobra zbieżność w strefie centralnej.
\end{itemize}

(Konsekwencja tego, że zadanie to jest {\itźle uwarunkowane}.)
\vspace{0.5cm}

Efekt Rungego -- rady praktyczne:

\begin{itemize}
\item zaczynać od interpolacji liniowej,

\item zwiększać liczbę węzłów i stopień wielomianu, aż do ustabilizowania istotnych miejsc.
\end{itemize}
\vspace{0.5cm}
Inne sposoby:
\begin{itemize}
\item interpolacja funkcjami sklejanymi
\item specjalny dobór węzłów.
\end{itemize}

\end{frame}
