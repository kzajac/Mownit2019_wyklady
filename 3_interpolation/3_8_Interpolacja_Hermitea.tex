\section{Interpolacja Hermite'a}
%\begin{frame}
%{3.8 Interpolacja Hermite'a}

%Więcej o metodzie i jej autorze \\
%(autor: Shayne Waldrom,University of Auckland):\\
%\vspace{5mm}
%https://www.math.auckland.ac.nz/$\sim$waldron/Hermite/he%rmite.html
%\end{frame}

%
\begin{frame}{Interpolacja Hermite'a}
\textcolor{blue}{Dane:}
\begin{itemize}
\item $k+1$ różnych węzłów: $x_{0}, x_{1}, \dots, x_{k}$

\item tzw. krotności węzłów dane liczbami naturalnymi $m_{0}, m_{1},\dots , m_{k}, \displaystyle \sum_{i=0}^{k}m_{i}=n+1$
\end{itemize}
\textcolor{blue}{Szukamy:}

Dla dowolnej funkcji $f$ -- szukamy wielomianu $H_{n}$ stopnia $\leq n,$ \\
takiego, że:

$H_{n}^{(j)}(x_{i})=f^{(j)}(x_{i})$ \quad dla $i=0, 1, \dots , k$ oraz
$j=0, 1, \dots, m_{i}$\\
\vspace{0.5cm}
Krotność mówi nam, ile pochodnych ma być równych.
Gdy $m_{i}=1$ -- interpolacja Lagrange'a.
\end{frame}

\begin{frame}{Rozwiązanie}
Suma krotności $i$ początkowych węzłów interpolacyjnych
\vspace{2mm}
$s(i)=\left\{\begin{array}{l}
0,\ i=0\\
m_{0}+m_{1}+\cdots+m_{i-1},\ i>0
\end{array}\right.$
\vspace{2mm}\\

Zauważmy, że każda liczba 
$0\leq l\leq n$ da się przedstawić w postaci $l=s(i)+j$ gdzie $0 \leq i \leq k$ oraz $0 \leq j \leq m_{i-1}$\\
\vspace{0.5cm}
Definiujemy wielomiany:\\
$p_{s(0)}(x)=1$

$p_{(s(i)+j)}(x)=(x-x_{0})^{m_{0}}(x-x_{1})^{m_{1}}\ldots(x-x_{i-1})^{m_{i-1}}(x-x_{i})^{j}(\star)$ \\
gdzie: $i=0, 1, \dots, k; \: j=0, 2, \dots , m_{i-1}$ \\
Wtedy szukany wielomian interpolacyjny to kombinacja linowa takich wielomianów (por. postać Newtona)
$$
H_{n}(x)=\sum_{l=0}^{n}b_{l}\cdot p_{l}(x)=\sum_{i=0}^{k}\sum_{j=0}^{m_{i-1}}b_{(s(i)+j)}\cdot p_{(s(i)+j)}(x)
$$
\end{frame}

\begin{frame}
Jak znaleźć współczynniki $b_{l}$:

$l$ - ustalone, $l=s(i)+j$
$$
H_{n}(x)=\underbrace{A(x)}_{b_{0},{b_{1},\ldots,b}_{l-1}}+b_{l}p_{l}(x)+\underbrace{B(..x)}_{b_{l+1},\ldots,b_{n}}(\star\star)
$$
gdzie:
$A(x)$ - kombinacja liniowa wielomianów $b_i$ o znanych już współczynnikach $b_{0},{b_{1},\ldots,b}_{l-1}$\\

\vspace{3mm}
$B(x)$ - kombinacja liniowa wielomianów $b_i$ o współczynnikach $b_{l+1},\ldots,b_{n}$
\end{frame}
\begin{frame}
Biorąc pod uwagę (*)
$$
p_{l}(x)=p_{(s(i)+j)}(x)=p_{s(i)}(x)(x-x_{i})^{j}
$$
oraz dla $m > l$ czyli wielomianów wchodzących w skład $B(x)$, każdy jest postaci
$$
p_{m}(x)=f_m(x)*(x-x_{i})^{k}(***)
$$
gdzie $k>j$ oraz $f_m(x)$ zawiera czynniki z(*) dla pozostałych węzłow. \\
\end{frame}
\begin{frame}
różniczkując $(\star\star)$

$$
H_{n}^{(j)}(x)=A^{(j)}(x)+b_{l}p^{(j)}_{s(i)}(x)(x-x_{i})^j+b_{l}p_{s(i)}(x)j!+B^{(j)}(x)
$$
Interesuje nas $H_{n}^{(j)}(x_{i})$\\
Ze względu na (***) $B^{(j)}(x_i)=0$ i mamy:
$$
H_{n}^{(j)}(x_{i})=A^{(j)}(x_{i})+b_{l}p_{s(i)}(x_{i})j!
$$
korzystając $\mathrm{z}$:
$$
H_{n}^{(j)}(x_{i})=f^{(j)}(x_{i})
$$
mamy:
$$
b_{l}=\frac{f^{(j)}(x_{i})-A^{(j)}(x_{i})}{p_{s(i)}(x_{i})j!}
$$

\end{frame}
