\section{B-splines - basic splines}
\begin{frame}{Funkcje B-sklejane}
    \begin{itemize}
        \item stosowane w grafice komputerowej do modelowania figur o skomplikowanych kształtach
        \item bazują na fakcie, że funkcje sklejane można wyrazić za pomocą kombinacji liniowej funkcji bazowych 
        \item takie funkcje bazowe nazywamy funkcjami B-sklejanymi (B-splines)
        \item dla danego zestawu  węzłów interpolacji - funkcje bazowe łatwo wyliczyć rekurencyjnie 
        \item algorytmy o dobrych własnościach numerycznych
    \end{itemize}
\end{frame}
\begin{frame}{Funkcje B-sklejane}
	$B_{j,k}(x)$ - B-spline rzędu k:
    \begin{itemize}
    \item  $B_{j,0}=
        \begin{cases}
            	1  &,\ x_{j} \leq x \leq x_{j+1}
            	\\
                0 &,\ \textrm{poza przedziałem}
        \end{cases}
        $
        \item wyższe rzędy $(k>0)$ - rekurencyjnie:
        \[
        	B_{j,k}(x)=\frac{x-x_{j}}{x_{j+k}-x_{j}}B_{j,k-1}(x)+\frac{x_{j+k+1}-x}
            {x_{j+k+1}-x_{j+1}}B_{j+1,k-1}(x)
        \]
        \item np. $B_{j,1}=\Psi_j$ (patrz interpolacja liniową funkcją sklejaną)
    \end{itemize}
    \begin{exampleblock}{Reprezentacja funkcji sklejanej stopnia k}
		\centering $S(x)=\sum_{j}p_{j}B_{j,k}(x)$
	\end{exampleblock}
	 gdzie $p_j$ - wspólczynniki (w grafice komputerowej są to tzw. zadane punkty kontrolne) 
\end{frame}
%%%%%%%%%%%%%%%%%%%%%%%%%%%%%%%%%%%%%%%%%%%%%%
\begin{frame}{Funkcje B-sklejane - własności}
     \begin{itemize}
        \item $B_{j,k}(x)>0, x\in[x_{j} \ x_{j+k+1}]$ 
        \item $B_{j,k}(x)=0, x\notin[x_{j} \ x_{j+k+1}]$ 
       
        \item w przedziale $[x_{j},\ x_{j+1}]$ istotne jest tylko $k+1$ funkcji : $B_{j-k,k}(x)\ldots  B_{j,k}(x)\neq 0$
		\item normalizacja: $\sum_{j}B_{j,k}(x)=\sum_{j=l-k}^{l}B_{j,k}(x)=1$ 
        \newline dla $x_{l}\leq x\leq x_{l+1}$
    \end{itemize}
\end{frame}
%%%%%%%%%%%%%%%%%%%%%%%%
\begin{frame}{Funkcje B-sklejane stopnia 3}
	\begin{figure}[h]
			\includegraphics[width=.57\linewidth]{img/4/spline_img_6}
	\end{figure}
    \[\underbrace{B_{j,3}(x)}_{b_j(x)}=
    	\begin{cases}
    		\frac{1}{6}z^{3} &,z=\frac{x-x_{j}}{d} \ \ x \in [x_{j},x_{j+1}] \\
            \frac{1}{6}[1+3(1+z(1-z))z] &,z=\frac{x-x_{j+1}}{d} \ x 
            \in [x_{j+1},x_{j+2}]\\
            \frac{1}{6}[1+3(1+z(1-z))(1-z)] &,z=\frac{x-x_{j+2}}{d} \ x 
            \in [x_{j+2},x_{j+3}] \\
            \frac{1}{6}(1-z)^{3} &,z=\frac{x-x_{j+3}}{d} \ x 
            \in [x_{j+3},x_{j+4}] \\
            0 \ &, x \notin [x_{j},x_{j+4}]
    	\end{cases}
    \]
\end{frame}





