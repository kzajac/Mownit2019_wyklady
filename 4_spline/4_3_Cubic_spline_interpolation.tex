\section{Interpolacja sześcienna (cubic spline)}
	\begin{frame}{Interpolacja sześcienna (cubic spine)}
	 W praktyce najczęściej używane.
		\begin{block}{}
			\begin{itemize}
				\item $s_{i}$ on $[x_{i},x_{i+1}] \rightarrow$ cubic polynomial:
                $\newline \ \ \ s_{i}(x)=a_{i}+b_{i}(x-x_{i})+c_{i}(x-x_{i})^{2}+
                d_{i}(x-x_{i})^{3}$
                \item $s_{i}(x_{i+1})=f(x_{i+1})$
                \item $s_{i}(x_{i+1})=s_{i+1}(x_{i+1})$
                \item $s^{'}_{i}(x_{i+1})=s^{'}_{i+1}(x_{i+1})$
                \item $s^{''}_{i}(x_{i+1})=s^{''}_{i+1}(x_{i+1})$
			\end{itemize}
		\end{block}
		
		
		Wypisując takie warunki dla każdego z "wewnętrznych"  punktów intepolacji i dodając warunki brzegowe można stworzyć układ równań i go rozwiązywać\\
		Istnieje jednak sprytniejsze podejście !
        
	\end{frame}