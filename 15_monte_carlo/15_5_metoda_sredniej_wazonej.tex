\section{Metoda średniej ważonej ({\it importance sampling})}
%%%%%%%%%%%%%%%%
\begin{frame}{Metoda średniej ważonej}
	Gdy $f(x)$ jest stała to w metodzie podstawowej - pojedynczy punkt to wynik dokładny \\
    Stąd wniosek: jeśli $f(x)$ gładkie to liczba losowań - mała
    \\[8pt]
    Metoda średniej ważonej 
     $I = \int_0^1 \frac{f(x)}{p(x)} p(x) dx = \int_0^1 g(x)p(x)dx$
    Chcemy wybrać takie $p(x)$ żeby:
    \begin{enumerate}
         \item $p(x)$ - ciągła, $x \in [0, 1]$  
    	\item $p(x) > 0, x \in [0, 1]$
        \item $\int_0^1 p(x) dx = 1$
        \item $\frac{f(x)}{p(x)}, x \in (0, 1)$ - znacznie gładsza niż $f(x)$
        \item $p(x)$ - dana prostym wzorem analitycznym
    \end{enumerate}
    
    %\[
    %	\left.
      %  	\begin{array}{ll}
      %  		\text{1, 2 - precyzyjne} \\
      %  		\text{3, 4 - rozmyte}
      %  	\end{array}
     %   \right\} 
      %  \text{warunki}
    %\]
\end{frame}
%%%%%%%%%%%%%%%%
%\begin{frame}{Metoda średniej ważonej - dygresja}
%	\begin{block}{Dygresja - sugestia, co do dobrego $g(x)$}
%		$X$ - zmienna losowa, \\
 %       $G(x)$ - dystrybuanta X; $G(x) = P(X < x)$
  %      \\[8pt]
   %     niech $G(x)$ - funkcja ściśle rosnąca: \[
     %   	P[\underbrace{G(x)}_\eta < \underbrace{G(x')}_{\eta'}] = P(x < x') = \underbrace{G(x')}_{\eta'}
    %    \]
        
    %    {\bf wniosek:} Jeżeli zmienna losowa $X$ ma ściśle rosnącą dystrybuantę $G(x)$, to $G(x)$ ma rozkład równomierny na $(0, 1)$
%	\end{block}
%\end{frame}
%%%%%%%%%%%%%%%%
\begin{frame}{Metoda średniej ważonej}
	%\begin{block}{Dygresja - sugestia, co do dobrego $g(x)$}
    %	... stąd - sposób obliczania $I = \int_0^1 f(x) dx$:
        \begin{enumerate}
        	\item wybieramy $p_1(x) > 0$ dla $x \in(0,1)$ - 1-sza propozycja,
            \item dobieramy stałą - $p(x) = \alpha p_1(x); \alpha \int_0^1 p_1(x) dx = 1$
            \item liczymy analitycznie dystybuantę $P(x) = \int_0^x p(x') dx'$
            \item             losujemy z rozkładem równomiernym: $y_1 \in (0, 1), i = 1, ..., N$
            \item stosujemy metodę odwrotnej dystybuanty $P^{-1}(y_i) = x_i \Rightarrow x_i, i = 1, ..., N$
            \item przybliżamy wartość całki: $I \approx \frac{1}{N} \sum_{i=1}^N g(x_i)=\frac{1}{N} \sum_{i=1}^N \frac{f(x_i)}{p(x_i)}$
        \end{enumerate}
%	\end{block}
\end{frame}
%%%%%%%%%%%%%%%%
%\begin{frame}{Metoda średniej ważonej - dygresja}
%	\begin{block}{Dygresja - sugestia, co do dobrego $g(x)$}
    %	\centering \includegraphics[width=.4\linewidth]{img/15/15_2_dobre_gx}
%	\end{block}
%\end{frame}