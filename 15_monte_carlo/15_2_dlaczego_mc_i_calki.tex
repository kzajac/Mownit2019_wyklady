\section{Dlaczego MC i całki?}
%%%%%%%%%%%%%%%%
\begin{frame}{Dlaczego MC i całki?}
	\begin{enumerate}
		\item[a)]
		Niech $y = g(x)$ to pewna zmienna losowa, której 
		 wartości  losujemy zgodnie 
        z rozkładem  ciągłym  $p(x)$ \\
           $x\in (a,b)$
            
            
            {\bf Wartość oczekiwana} $Y: E\{Y\} = \int_a^b g(x) p(x) dx$ \hfill $(*)$
	\end{enumerate}
\end{frame}
%%%%%%%%%%%%%%%%
\begin{frame}{Dlaczego MC i całki?}
	\begin{enumerate}
		\item[b)]
			chcemy obliczyć całkę: $I = \int_a^b f(x) dx$ \\
            niech $p: p(x) > 0$ dla $x \in (a, b), \int_a^b p(x) dx = 1$ \\
            $p(x)$ - spełnia warunki, aby być gęstością rozkładu pewnej zmiennej losowej przyjmującej wartości z $(a, b)$ \\
            $I = \int_a^b \frac{f(x)}{p(x)} p(x) dx = \int_a^b g(x)p(x)dx$ \hfill $\leftarrow$ całka postaci $(*)$ \\
            Obliczanie całki można zawsze przedstawić jako zagadnienie obliczania wartości oczekiwanej pewnej zmiennej losowej ciągłej.
            \\[8pt]
            (suma szeregu - zmienna losowa dyskretna)
	\end{enumerate}
\end{frame}