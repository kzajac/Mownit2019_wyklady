\section{Gaussian elimination-LU factorization}

\begin{frame}{Eliminacja Gaussa - jeszcze raz}
1 etap:
$$
\left(\begin{array}{lll}
a_{11} & a_{12} & a_{13}\\
0 & a_{22}^{(2)} &a_{23}^{(2)}\\
0 & a_{32}^{(2)} & a_{33}^{(2)}
\end{array}\right) \cdot
 \left(\begin{array}{l}
x_{1}\\
x_{2}\\
x_{3}
\end{array}\right)=
\left(\begin{array}{l}
b_{1} \\
b_{2}^{(2)} \\
b_{3}^{(2)}
\end{array}\right)
$$
2 -etap:
$$
U = \left(\begin{array}{lll}
a_{11} & a_{12} & a_{13}\\
0 & a_{22}^{(2)} &a_{23}^{(2)}\\
0 & 0 & a_{33}^{(3)}
\end{array}\right)
,\ c=\left(\begin{array}{l}
b_{1}\\
b_{2}^{(2)}\\
b_{3}^{(3)}
\end{array}\right)
$$
\end{frame}
\begin{frame}{}

Ogólnie po $k$ etapach otrzymujemy:
$$
A^{(k+1)}\cdot x=b^{(k+1)}
$$

$a_{ij}^{(k+1)}=a_{ij}^{(k)}-\displaystyle \frac{a_{ik}^{(k)}}{a_{kk}^{(k)}}\cdot a_{kj}^{(k)}, i, j>k$ \fbox{1a}


$b_{i}^{(k+1)}=b_{i}^{(k)}-\displaystyle \frac{a_{ik}^{(k)}}{a_{kk}^{(k)}}\cdot b_{k}^{(k)}, i>k$ \fbox{1b}

przy założeniu, że $a_{kk}^{(k)}\neq 0$ (w przeciwnym przypadku $\rightarrow$ zamiana wierszy)

\end{frame}
\begin{frame}{Gaussian elimination}
W efekcie $A^{(n)}=U$ - trójkątna górna (upper triangular) \\
wprowadzamy oznaczenie: $l_{ik}=\displaystyle \frac{a_{ik}^{(k)}}{a_{kk}^{(k)}}, \quad a_{kk}^{(k)}\neq 0 $
\end{frame}
\begin{frame}{Metoda Gaussa a faktoryzacja LU}
Jeśli oznaczymy:
$$
L^{(k)}=\begin{bmatrix}
1 \\
 & \ddots & & & 0 \\
 & & 1\\
 & &   -l_{k+1,k}  & \ddots\\
 0 & &  \vdots &\\
 & & -l_{n,k} &  0  & & 1

\end{bmatrix}
$$
Jest to macierz trójkątna dolna.

\fbox{1a} możemy zapisać w postaci macierzowej
$$
A^{(k+1)}=L^{(k)}\cdot A^{(k)}
$$



\end{frame}
\begin{frame}{Metoda Gaussa a faktoryzacja LU}
i w konsekwencji
$$
U=A^{(n)}=L^{(n-1)}\cdot L^{(n-2)}\cdot\ldots\cdot L^{(1)}\cdot A \ \ \ (*)
$$

łatwo pokazać, że:
$$
(L^{(k)})^{-1}=\begin{bmatrix}
1 \\
 & \ddots & & & 0 \\
 & & 1\\
 & &   l_{k+1,k}  & \ddots\\
 0 & &  \vdots \\
 & & l_{n,k} &  0  & & 1

\end{bmatrix}
$$
bo $L^{(k)}\cdot(L^{(k)})^{-1}=I$
\end{frame}
\begin{frame}{Metoda Gaussa a faktoryzacja LU}
Mnożąc (*) kolejno przez $(L^{(n-1)})^{-1}, (L^{(n-2)})^{-1}$, . . . , $(L^{(1)})^{-1}$ otrzymamy:
$$
A=\underbrace{(L^{(1)})^{-1}\cdot(L^{(2)})^{-1}\cdot\ldots\cdot(L^{(n-1)})^{-1}}\cdot U\rightarrow A=L\cdot U
$$
\hspace{17mm}
L=
$
\begin{bmatrix}
1 & & & \\
l_{21}  & 1 & & 0 \\
l_{31} & l_{32} & \ddots & 0 \\
\vdots & \vdots & l_{k+1,k} & \vdots \\
l_{n1} & l_{n2} & l_{nk} & 1
\end{bmatrix}
$
\end{frame}
\begin{frame}{Metoda Gaussa a faktoryzacja LU}
Na każdym etapie macierze L oraz U są przechowywane w spakowanej postaci:\\
\begin{itemize}
    \item $\bullet$~do przechowywania $A^{(1)}, A^{(2)}$, . . . wystarczy jedna macierz 
    \item $\bullet~ l_{ij}$ są przechowywanie w miejscu powstających kolumn z zerami
    \item $\bullet$~ nie musimy pamiętać "1" na diagonali
\end{itemize}

 \newline

\end{frame}