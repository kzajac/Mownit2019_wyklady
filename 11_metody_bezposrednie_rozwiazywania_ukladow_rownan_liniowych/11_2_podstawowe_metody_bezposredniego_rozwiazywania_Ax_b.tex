\section{Podstawowe metody bezpośredniego rozwiązywania Ax=b}
\begin{frame}{Wstęp}
\begin{block}{Układy równań liniowych}
$$
E_{1}:a_{11}x_{1}+a_{12}x_{2}+\cdots+a_{1n}x_{n}=b_{1}
$$
$$
\vdots \hspace{55.5mm} \vdots
$$
$$
E_{n}:a_{n1}x_{1}+a_{n2}x_{2}+\cdots+a_{nn}x_{n}=b_{n}
$$
\end{block}
Podstawą metod bezpośrednich  są operacje elementarne, które nie zmieniają rozwiązań układów równań.


\end{frame}
%%%%%%%%%%%%%%%%%%%%%%%%%%%%%%%%%
\begin{frame}{Operacje elementarne}

Operacje elementarne: \newline
\begin{itemize}
    \item $\bullet$  $(\lambda E_{i})\rightarrow E_{i}$ Pomnożenie (podzielenie) przez dowolną niezerową liczbę dowolnie wybranego wiersza macierzy.  
\item $\bullet$ $(E_{i}+\lambda E_{j})\rightarrow E_{i}$ Dodanie do dowolnie wybranego wiersza macierzy wielokrotności innego wiersza  

\item $\bullet$ $E_{i}\leftrightarrow E_{j}$ 
Zamiana miejscami dwóch dowolnych wierszy macierzy. 
\end{itemize}
\newline
gdzie $\lambda$-stała różna od zera \newline
\end{frame}
\begin{frame}
Celem jest uzyskanie postaci trójkątnej: (triangular or reduced form)
\begin{flushright}
$a_{11}x_{1}+a_{12}x_{2}+\cdots+a_{1n}x_{n}=b_{1} \linebreak$
$0+a_{22}'x_{2}+\cdots+a_{2n}'x_{n}=b_{2}'
 \linebreak$
$ \hspace{60mm} \vdots \linebreak$
$0+\cdots+0+a_{nn}'x_{n}=b_{n}' \linebreak$
\end{flushright}
i wtedy ostanią niewiadomą możemy wyliczyć wprost:

$$ x_{n}=  \frac{b_{n}'}{a_{nn}'}$$
 A potem wszystkie pozostałe poprzez podstawianie wsteczne(backward substitution)
$$x_{i}= \frac{1}{a_{ii}'}(b_{i}'-\displaystyle \sum_{j=i+1}^{n} a_{ij}'x_{j})$$ 
\end{frame}
%%%%%%%%%%%%%%%%%%%%%%%%%%%%%%%%%
\begin{frame}{Gaussian elimination}
\scalebox{0.92}{
$\left(\begin{array}{lll}
a_{11} & a_{12} & a_{13}\\
a_{21} & a_{22} & a_{23}\\
a_{31} & a_{32} & a_{33}
\end{array}\right)\cdot \left(\begin{array}{l}
x_{1}\\
x_{2}\\
x_{3}
\end{array}\right)=\left(\begin{array}{l}
b_{1}\\
b_{2}\\
b_{3}
\end{array}\right)
\begin{array}{l}
\frac{a_{21}}{a_{11}} 
\\
\\
\\
\end{array}
\begin{array}{l}
\bigg\}-
\\
\\
\end{array}
\left. \begin{array}{l}
\frac{a_{31}}{a_{11}}
\\
\\
\\
\end{array}
\right\}
- 
$}
$$$
Aby to zadziałało $a_{11}\neq 0$.\\
Po pierwszym etapie mamy:
%\scalebox{0.90}{
$$
\left(\begin{array}{lll}
a_{11} & a_{12} & a_{13}\\
0 & a_{22}^{(2)} &a_{23}^{(2)}\\
0 & a_{32}^{(2)} & a_{33}^{(2)}
\end{array}\right) \cdot
 \left(\begin{array}{l}
x_{1}\\
x_{2}\\
x_{3}
\end{array}\right)=
\left(\begin{array}{l}
b_{1} \\
b_{2}^{(2)} \\
b_{3}^{(2)}
\end{array}\right)
$$
%}
$$a_{22}^{(2)}=a_{22}-\frac{a_{21}}{a_{11}}\cdot a_{12}, a_{23}^{(2)}=\ldots,$$
$$ a_{32}^{(2)}=a_{32}-\frac{a_{31}}{a_{11}}\cdot a_{12}, a_{33}^{(2)}=\ldots,$$

$$
b_{2}^{(2)}=b_{2}-\frac{a_{21}}{a_{11}}\cdot b_{1},\ b_{3}^{(2)}=b_{3}-\frac{a_{31}}{a_{11}}\cdot b_{1},
$$


\end{frame}
\begin{frame}{Gaussian elimination}
W wyniku następnego etapu otrzymamy:
\scalebox{0.92}{
$\left(\begin{array}{lll}
a_{11} & a_{12} & a_{13}\\
0 & a_{22}^{(2)} &a_{23}^{(2)}\\
0 & a_{32}^{(2)} & a_{33}^{(2)}
\end{array}\right)\cdot \left(\begin{array}{l}
x_{1}\\
x_{2}\\
x_{3}
\end{array}\right)=\left(\begin{array}{l}
b_{1} \\
b_{2}^{(2)} \\
b_{3}^{(2)}
\end{array}\right)
\begin{array}{l}
\\
\frac{a_{32}^{(2)}}{a_{22}^{(2)}} 
\\
\\
\end{array}
\begin{array}{l}
\\
\bigg\}-
\\
\end{array}
$}
$$
U = \left(\begin{array}{lll}
a_{11} & a_{12} & a_{13}\\
0 & a_{22}^{(2)} &a_{23}^{(2)}\\
0 & 0 & a_{33}^{(3)}
\end{array}\right)
,\ c=\left(\begin{array}{l}
b_{1}\\
b_{2}^{(2)}\\
b_{3}^{(3)}
\end{array}\right)
$$
$$ a_{33}^{(3)}=a_{33}^{(2)}-\frac{a_{32}^{(2)}}{a_{22}^{(2)}}\cdot a_{23}^{(2)} $$
$$

$$
b_{3}^{(3)}=b_{3}^{(2)}-\displaystyle \frac{a_{32}^{(2)}}{a_{22}^{(2)}}\cdot b_{2}^{(2)}
$$
\end{frame}
\begin{frame}
Ogólnie po $k$ etapach otrzymujemy:

$
A^{(k+1)}\cdot x=b^{(k+1)}\cdot x
$
$$
$$
$a_{ij}^{(k+1)}=a_{ij}^{(k)}-\displaystyle \frac{a_{ik}^{(k)}}{a_{kk}^{(k)}}\cdot a_{kj}^{(k)}, i, j>k$ \fbox{1a}
$$
$$
$b_{i}^{(k+1)}=b_{i}^{(k)}-\displaystyle \frac{a_{ik}^{(k)}}{a_{kk}^{(k)}}\cdot b_{k}^{(k)}, i>k$ \fbox{1b}

przy założeniu, że $a_{kk}^{(k)}\neq 0$\\
Jeśli $a_{kk}^{(k)}= 0$ należy przestawic wiersze. \newline

\vspace{0.5cm}

W efekcie $A^{(n)}=U$ staje się macierzą trójkątną górną  (upper triangular) \\
\end{frame}
%\begin{frame}{Gaussian elimination}
%Jeśli $a_{kk}^{(k)}= 0$ należy przestawic %wiersze \newline W efekcie $A^{(n)}=U$ - upper %triangular \\
%multiplier $l_{ik}=\displaystyle %\frac{a_{ik}^{(k)}}{a_{kk}^{(k)}}, \quad %a_{kk}^{(k)}\neq 0 $
%\end{frame}
\begin{frame}{Gaussian elimination with backward substitution}
\begin{exampleblock}{Elimination}
for $\mathrm{i}:=1$ to n-1 do

\hspace{4mm} begin

\hspace{7mm}
$\mathrm{p}:=$smallest integer $in[i,\ n]$ : $a_{pi}\neq 0$;  \newline
\hspace*{7mm} if no $\mathrm{p}$ then no unique solution exist!; STOP;

\hspace{7mm} if $p\neq i$ then $E_{p}\leftrightarrow E_{i}$ /przestawienie/

\hspace{7mm} for $\mathrm{j}:=\mathrm{i}+1$ to $\mathrm{n}$ do $(E_{j}-\displaystyle \frac{a_{ji}}{a_{ii}}E_i\rightarrow E_{j})$

\hspace{4mm} end

if $a_{nn}=0$ then no unique solution exist!; STOP;
    	\end{exampleblock}

\end{frame}
%%%%%%%%%%%%%%%%%%%%%%%%%%%%%%%%%
\begin{frame}{Gaussian elimination with backward substitution}
\begin{exampleblock}{Backward substitution}
$x_{n}=b_{n}'/a_{nn}'$;

for $\mathrm{i}:=\mathrm{n}-1$ downto 1 do $x_{i}=(b_{i}-\displaystyle \sum_{j=i+1}^{n}a_{ij}x_{j})/a_{ii}$;
\end{exampleblock}

\textbf{Złożoność obliczeniowa}
$O(n^3)$

Liczba działań mnożenia $\mathrm{i}$ dzielenia wynosi 
$$\underbrace{\sum_{i=1}^{n-1}}_{\text{petla zewnetrzna}}
(\underbrace{(n-i)}_{\substack{ \text{petla}\\ \text{wewnetrzna}}}
(\underbrace{1}_{\text{dzielenie}\frac{a_{ji}}{a_{ii}}}
+\underbrace{n-i}_{\substack{\text{mnozenie }\\ a \text{ w }  E_i}}
+\underbrace{1}_{\substack{\text{mnozenie}  \\  b_i}}))+\underbrace{\frac{1}{2}n(n-1)}_{\substack{\text{podstawianie}  \\ \text{wsteczne}}}=$$

$\displaystyle =\frac{1}{3}\cdot(n^{3}+3n^{2}-n)$

Liczba działań dodawania $\mathrm{i}$ odejmowania wynosi
$\frac{1}{6}\cdot(2n^{3}+3n^{2}-5n)$




\end{frame}
%%%%%%%%%%%%%%%%%%%%%%%%%%%%%%%%%
%\begin{frame}{Gaussian elimination with backward substitution}
%\begin{itemize}
%\item \textbf{Macierz odwrotna} \newline
%$A^{-1}$ : $Ax=Ib\rightarrow Ix=A^{-1}b$
%\begin{flushright}
%{\it Zadanie}: Ułóż algorytm
%\end{flushright}
%\item \textbf{Wyznacznik macierzy}
%$\det A^{-1}\quad(A=LU;\quad \det A=\det L\cdot\det U,\ \cdots)$
%\begin{flushright}
%{\it Zadanie}: Sprawdzić
%\end{flushright}
%\item \textbf{Macierz Gaussa-Jordana}
%$$
%A'=\left(\begin{array}{llll}
%a_{11}' & 0 & \cdots & 0\\
%0 & a_{22}' & \cdots & 0\\
%\vdots &  &  & \vdots\\
% 0 & \cdots & 0 & a_{nn}'
%\end{array}\right)
%$$
%\begin{flushright}
%{\it Zadanie}: Sprawdzić %złożoność, algorytm
%\end{flushright}
%\end{itemize}
%\end{frame}
%%%%%%%%%%%%%%%%%%%%%%%%%%%%%%%%%
\begin{frame}{O wyborze elementów podstawowych - pivots}
$$
\left(\begin{array}{lll}
10 & -7 & 0\\
-3 & 2.099 & 6\\
5 & -1 & 5
\end{array}\right)\ \cdot
\left(\begin{array}{l}
x_{1}\\
x_{2}\\
x_{3}
\end{array}\right)=\left(\begin{array}{l}
7\\
3.901\\
6
\end{array}\right)\ 
%\begin{array}{l}
%(:10)\\
%(\cdot(-3))\\
%(\cdot(-5))
%\end{array}
$$
rozwiązanie:
$$
x=\left(\begin{array}{l}
0\\
-1\\
1
\end{array}\right)
$$
Po "wyzerowaniu"  pierwszej kolumny (arytmetyka {\it fl} z 5-cioma cyframi znaczącymi)
$$
\left(\begin{array}{lll}
10 & -7 & 0\\
0 & -0.001 & 6\\
0 & 2.5 & 5
\end{array}\right)\ \cdot
\left(\begin{array}{l}
x_{1}\\
x_{2}\\
x_{3}
\end{array}\right)=\left(\begin{array}{l}
7\\
6.001\\
2.5
\end{array}\right)\ 
\begin{array}{l}
\\
\\
\\
\end{array}
$$
\fbox{$-0.001$}
 -bardzo mały w porównaniu z pozostałymi 

\end{frame}
%%%%%%%%%%%%%%%%%%%%%%%%%%%%%%%%%
\begin{frame}{O wyborze elementów podstawowych - pivots}
 $E_{3}\leftarrow E_{3}-\frac{2.5}{-0.001}\cdot E_{2}=E_{3}+2.5\cdot 10^3 \cdot E_{2}$ :
\begin{center}
$b_3^{(3)}=2.5+2.5 \cdot 10^{3}\cdot 6.001 =2.5+ \underbrace{1.50025}_{\text{wybieram 5 cyft znaczących}}\cdot 10^{4} \approx 2.5+ 1.5002\cdot 10^{4}$

$$
a_{33}^{(3)}\cdot x_3=b_3^{(3)}
$$

$$
\underbrace{(5 + 2.5\cdot 10^{3} \cdot 6)}_{15005}\cdot x_{3}= 2.5 + 1.5002\cdot 10^{4}
$$
$$
15005\cdot\ x_{3}=15004\rightarrow x_{3}=\frac{15004}{15005}=0.99993
$$
\end{center}
Prawdziwe $x_{3}=1$ - błąd wydaje się mały!
\newline
\end{frame}
%%%%%%%%%%%%%%%%%%%%%%%%%%%%%%%%%
\begin{frame}{O wyborze elementów podstawowych - pivots}
Ale z równania 2) :
$$
-0.001\cdot x_{2}+6\cdot 0.99993=6.001
$$
$$
x_{2}=\frac{6.001-5.9995}{0.001}=\underline{-1.5}
$$

Z równania 1) :
$$
10\cdot\ x_{1}+(-7)\cdot(-1.5)+0=7\rightarrow x_{1}=\underline{-0.35}
$$
Zamiast (0,-1,1) mamy $(-0.35,\ -1.5,0.99993)$ - Dlaczego? \newline 
Przecież nie było akumulacji błędów, a macierz nie jest bliska osobliwej.
\begin{alertblock}{}
Powodem błędu jest zbyt mały element wiodący (pivot). Poszczególne wiersze dzielimy przez niego (czyli mnożymy przez dużą liczbę), więc błędy zaokrągleń stają sie duże w stosunku do współczynników oryginalnej macierzy!
\end{alertblock}
\begin{flushright}
\textit{Zadanie}: Sprawdzić co będzie po zamianie 2) i 3) 
\end{flushright}
\end{frame}
%%%%%%%%%%%%%%%%%%%%%%%%%%%%%%%%%
\begin{frame}{ O wyborze elementów podstawowych - pivots}

Ogólna zasada: jeżeli w poszczególnych krokach mnożymy przez liczbę mniejszą lub równą 1, to możemy oczekiwać rozwiązania dokładnego.
\begin{exampleblock}{Partial pivoting:}
W $\mathrm{k}$-tym kroku wybieramy wiersz o największym współczynniku w nieredukowalnej części w $\mathrm{k}$-tej kolumnie

\end{exampleblock}

\begin{exampleblock}{Skalowanie - równoważenie (equalibration) macierzy}
Maksymalne elementy w każdej kolumnie i każdym wierszu - tego samego rzędu
\end{exampleblock}

\end{frame}
%%%%%%%%%%%%%%%%%%%%%%%%%%%%%%%%%
\begin{frame}{ O wyborze elementów podstawowych - pivots}

Np. Macierz $A$ można zrównoważyć na dwa sposoby (dzieląc przez $10^9$ pierwszą kolumnę albo drugi i trzeci wiersz):

$$A=\left(\begin{array}{lll}
1 & 1 & 1\\
10^{9} & -1 & 1\\
10^{9} & 1 & 0
\end{array}\right)
$$
$$
B=\left(\begin{array}{lll}
10^{-9} & 1 & 1\\
1 & -1 & 1\\
1 & 1 & 0
\end{array}\right)
$$
$$
C=\left(\begin{array}{lll}
1 & 1 & 1\\
1 & -10^{-9} & -10^{-9}\\
1 & -10^{-9} & 0
\end{array}\right)
$$

Macierz $B$ zachowanie poprawne

Macierz $C$ da macierz osobliwą na maszynie o mniej niż $ 9$-ciu cyfrach znaczących

\end{frame}

