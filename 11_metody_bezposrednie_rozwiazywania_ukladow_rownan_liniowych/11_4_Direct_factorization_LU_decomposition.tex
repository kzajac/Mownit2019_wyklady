
\section{Faktoryzacja LU}
\begin{frame}{Wprowadzenie}
Zaleta: mając $A=L\cdot U\rightarrow $ dla dowolnego $b$
$$
A\cdot x=b
$$
Czyli:
$$
L\cdot(U\cdot \mathrm{x})=b
$$
Możemy więc rozwiązać dwa równania, ale z macierzami trójkątnymi (podstawianie wsteczne):
$$
L\cdot z=b\ (*)
$$
$$
U\cdot x=z\ (**)
$$
\begin{center}
$(*)z_{1}=\displaystyle \frac{b_{1}}{l_{11}};z_{i}=\frac{1}{l_{ii}}\cdot(b_{i}-\sum_{j=1}^{i-1}l_{ij}\cdot z_{j})$ , $i=2$, 3, . . . , $n$
\end{center}
$(**)x_{n}=\displaystyle \frac{z_{n}}{u_{nn}};x_{i}=\frac{1}{u_{ii}}\cdot(z_{i}-\sum_{j=i+1}^{n}u_{ij}\cdot x_{j})$ , $i=n-1, n-2$, . . . , 1
\end{frame}
\begin{frame}{Wprowadzenie}
Faktoryzacja LU w ogólności polega na znalezieniu $l$ oraz $u$ takich, że:
\scalebox{0.85}{
$\left(\begin{array}{llll}
a_{11} & a_{12} & \cdots & a_{1n}\\
a_{21} & a_{22} & \cdots & a_{2n}\\
\vdots &  &  & \\
a_{n1} & a_{n2} & \cdots & a_{nn}
\end{array}\right)= \left(\begin{array}{lllll}
l_{11} & 0 & \cdots & & 0\\
l_{21} & l_{22} & 0 &\cdots & 0\\
\vdots &  & \ddots & \\
l_{n1} & l_{n2} & \cdots & & l_{nn}
\end{array}\right) \cdot \left(\begin{array}{llll}
u_{11} & u_{12} & \cdots & u_{1n}\\
0 & u_{22} & \cdots & u_{2n}\\
\vdots &  & \ddots & \\
0 & \cdots & 0 & u_{nn}
\end{array}\right)$
}
\vspace{0.5cm}

\# niewiadomych: $n^{2}+n$ (bo dwa razy przekątna) \newline
\# r\'{o}wna\'{n}: $a_{ij}=\displaystyle \sum_{k=1}^{n}l_{ik}\cdot u_{kj}\rightarrow n^{2}$
\end{frame}
\begin{frame}{Wprowadzenie}
Potrzeba dodatkowego warunku dla przekątnej -
stosuje się różne wersje warunków dla faktoryzacji LU :

$\bullet$ Doolittle: $l_{ii}=1$

$\bullet$ Crout: $u_{ii}=1$

$\bullet$ Choleski (stosowany dla macierzy symetrycznej i dodatnio określonej): $l_{ii}=u_{ii}$

%Ad.Choleski - macierz symetryczna, dodatkowo określona $A=(LD^{\frac{1}{2}})\cdot(D^{\frac{1}{2}}L^{T})$ 
\newline
\vspace{0.5 cm}

Sposób: liczymy na przemian - wiersze, kolumny, wiersze, kolumny ... w A.

Po obliczeniu - L,U zapisane w A.
\end{frame}

\begin{frame}{Faktoryzacja Doolittle'a}
LU inaczej - lepszy wgląd w kolejność obliczeń
\scalebox{0.90}{
$\begin{bmatrix}
(1) & 0 & \cdots & 0 & 0\\
l_{21} & (1) &  &  & \\
\vdots &  &  &  & \\
l_{i1} & l_{i2} & \cdots & (1) & 0\\
\vdots &  &  &  & \\
l_{n1} & l_{n2} & \cdots & l_{n,n-1} & (1)
\end{bmatrix} \cdot \begin{bmatrix}
u_{11} & u_{12} & \cdots & u_{1n}\\
0 & u_{22} & \cdots & u_{2n}\\
 &  &  & \vdots\\
 &  & u_{ii} & u_{in}\\
 &  &  & \vdots\\
 0 & \cdots & \cdots & u_{nn}
\end{bmatrix}=\begin{bmatrix}
a_{11} & a_{12} & \cdots & a_{1n}\\
a_{21} & a_{22} & \cdots & a_{2n}\\
 &  &  & \vdots\\
 a_{i1} & a_{i2} & \cdots & a_{in}\\
 &  &  & \vdots\\
 a_{n1} & a_{n2} & \cdots & a_{nn}
\end{bmatrix}$
}

\newline
\\
W $\mathrm{i}$-tym wierszu macierzy $\mathrm{L}$ różne od $0$ tylko pierwsze $\mathrm{i}$ element\'{o}w $W\mathrm{j}$-tej kolumnie macierzy $\mathrm{U}$ różne od $0$ tylko pierwsze $\mathrm{j}$ elementów

\end{frame}

\begin{frame}{Faktoryzacja Doolittle'a}
Dla $i\leq j$ np. $i=1$, $j=1$
\scalebox{0.90}{
$\begin{bmatrix}
\colorbox{red}{(1)} & \colorbox{red}{0} & \cdots& \colorbox{red}{0} & \colorbox{red}{0}\\
l_{21} & (1) &  &  & \\
\vdots &  &  &  & \\
l_{i1} & l_{i2} & \cdots & (1) & 0\\
\vdots &  &  &  & \\
l_{n1} & l_{n2} & \cdots & l_{n,n-1} & (1)
\end{bmatrix} \cdot \begin{bmatrix}
\colorbox{red}{u}_{11} & \colorbox{blue}{u}_{12} & \cdots & \colorbox{blue}{u}_{1n}\\
\colorbox{red}{0} & u_{22} & \cdots & u_{2n}\\
 &  &  & \vdots\\
  &  & u_{ii} & u_{in}\\
 \colorbox{red}{0} &  &  & \vdots\\
 \colorbox{red}{0} & \cdots & \cdots & u_{nn}
\end{bmatrix}=\begin{bmatrix}
\colorbox{red}{a}_{11} & a_{12} & \cdots & a_{1n}\\
a_{21} & a_{22} & \cdots & a_{2n}\\
 &  &  & \vdots\\
 a_{i1} & a_{i2} & \cdots & a_{in}\\
 &  &  & \vdots\\
 a_{n1} & a_{n2} & \cdots & a_{nn}
\end{bmatrix}$

}
\newline
\\

$a_{1j}= \sum_{p=1}^{n}l_{1p}\cdot u_{pj}$
ale: $l_{11}=1$,  $l_{1p}=0$ dla $p>1$, więc:

$\colorbox{red}{u}_{11}=a_{11}$\\
podobnie dla wszystkich $j>1$:
$\colorbox{blue}{u}_{1j}=a_{1j}$



\end{frame}
\begin{frame}{Faktoryzacja Doolittle'a}
Dla $i>j$ np. $j=1$, $i=2$

    \scalebox{0.90}{
$\begin{bmatrix}
(1) & 0 & \cdots & 0 & 0\\
\colorbox{red}{l}_{21} & \colorbox{red}{(1)} &  &  &\colorbox{red}{0} \\
\vdots &  &  &  & \\
\colorbox{blue}{l}_{i1} & l_{i2} & \cdots & (1) & 0\\
\vdots &  &  &  & \\
\colorbox{blue}{l}_{n1} & l_{n2} & \cdots & l_{n,n-1} & (1)
\end{bmatrix} \cdot \begin{bmatrix}
\colorbox{red}{u}_{11} & u_{12} & \cdots & u_{1n}\\
\colorbox{red}{0} & u_{22} & \cdots & u_{2n}\\
 &  &  & \vdots\\
 &  & u_{ii} & u_{in}\\
 &  &  & \vdots\\
 \colorbox{red}{0} & \cdots & \cdots & u_{nn}
\end{bmatrix}=\begin{bmatrix}
a_{11} & a_{12} & \cdots & a_{1n}\\
\colorbox{red}{a}_{21} & a_{22} & \cdots & a_{2n}\\
 &  &  & \vdots\\
 a_{i1} & a_{i2} & \cdots & a_{in}\\
 &  &  & \vdots\\
 a_{n1} & a_{n2} & \cdots & a_{nn}
\end{bmatrix}$
}

\vspace{0.5cm}
$l_{i1}\cdot u_{11}=a_{i1}$

$\colorbox{red}{l}_{21}=\displaystyle \frac{a_{21}}{u_{11}}$

I ogólnie(dla $j=1$):
$\colorbox{blue}{l}_{i1}=\displaystyle \frac{a_{i1}}{u_{11}}$\\
W kolejnych  krokach analogicznie -  mamy już wyliczone $k-1$ wierszy $u$ i k-1 kolumn $l$ i z nich korzystamy. 
\end{frame}

\begin{frame}{Faktoryzacja Doolittle'a}
W kroku $\mathrm{k}$ wyznaczamy $U_{k,k}, U_{k,k+1}$, . . . , $U_{k,n}$
oraz (można równolegle) $L_{k+1,k}$, . . . , $L_{n,k}$ \\
Ogólnie - ze wzoru na mnożenie macierzy i z faktu, że $l_{ip}=0$ dla $p>i$ oraz $u_{pj}=0$ dla $p>j$\\
\scalebox{0.85}{
$a_{ij}= \displaystyle \sum_{p=1}^{\min(i,j)} l_{ip}\cdot u_{pj}, \ \  i, \ \  j=1$, 2, . . . , $n$ 
}\\

\begin{tabular}{|c|c|}
\hline
 $\mathrm{i}\leq \mathrm{j}$   & $ \mathrm{i}>\mathrm{j}$\\
  \hline\\
  $a_{ij}=\displaystyle \sum_{p=1}^{i-1}l_{ip}\cdot u_{pj}+l_{ii}\cdot u_{ij},  & $a_{ij}=\displaystyle \sum_{p=1}^{j-1}l_{ip}\cdot u_{pj}+l_{ij}\cdot u_{jj}\\
  $u_{ij}=a_{ij}- \displaystyle \sum_{p=1}^{i-1}l_{ip}\cdot u_{pj}$ &$l_{ij}=\displaystyle \frac{1}{u_{jj}}(a_{ij}-\sum_{p=1}^{j-1}l_{ip}\cdot u_{pj})$ \\
  \hline
\end{tabular}




$L_{ij}, U_{ij}$ przechowywane w $\mathrm{L}/\mathrm{U}$ (bez "1" $\mathrm{z}$ diagonali)
\end{frame}
\begin{frame}{Związek z eliminacją Gaussa}


Eliminacja Gaussa - wielokrotne pobieranie, poprawianie $a_{i,j}$


$$
a_{ij}^{(k+1)}\ =\ a_{ij}^{(k)} -a_{ik}^{(k)}\ \cdot \underbrace{\frac{a_{kj^{(k)}}}{a_{kk}^{(k)}}}_{l_{kj}}
$$

$$u_{k,j}=a_{k,j}^{(k)}$$
Różnica $\rightarrow$ kumulacja w 1 kroku

\end{frame}
\begin{frame}{Faktoryzacja Crouta}
\scalebox{0.90}{
$
\begin{bmatrix}
l_{11} & 0 & \cdots & 0 &  0 \\
l_{21} & l_{22}  \\
\vdots & \vdots & \ddots \\
l_{i1} & l_{i2}  & \cdots \\
\vdots & \vdots\\
l_{n1} & l_{n2} & \cdots & l_{n,n-1} & l_{nn}
\end{bmatrix} \cdot
\begin{bmatrix}
1 & u_{12} & \cdots & u_{1n}\\
0 & 1 & \cdots & u_{2n}\\
\vdots & 0 &  & \vdots\\
 & \vdots &  & u_{in}\\
 &  &  & \vdots\\
 0 & \cdots & & 1
\end{bmatrix}
=
\begin{bmatrix}
a_{11} & a_{12} & \cdots & a_{1n}\\
a_{21} & a_{22} & \cdots & a_{2n}\\
 \vdots & \vdots & & \vdots\\
 a_{i1} & a_{i2} & \cdots & a_{in}\\
 \vdots & \vdots & &\vdots\\
 a_{n1} & a_{n2} & \cdots & a_{nn}
\end{bmatrix}
$
}

\end{frame}
\begin{frame}{Faktoryzacja Crouta}
Analogicznie dla faktoryzacji Doolitla:
- ze wzoru na mnożenie macierzy i z faktu, że $l_{ip}=0$ dla $p>i$ oraz $u_{pj}=0$ dla $p>j$\\
\scalebox{0.85}{
$a_{ij}= \displaystyle \sum_{p=1}^{\min(i,j)} l_{ip}\cdot u_{pj}, \ \  i, \ \  j=1$, 2, . . . , $n$ 
}\\

\begin{tabular}{|c|c|}
\hline
 $\mathrm{i}\geq \mathrm{j}$   & $ \mathrm{i}<\mathrm{j}$\\
  \hline
   $a_{ij}=\displaystyle \sum_{p=1}^{j-1}l_{ip}\cdot u_{pj}+l_{ij}\cdot u_{jj},$& $a_{ij}= \displaystyle \sum_{p=1}^{i-1}l_{ip}\cdot u_{pj}+l_{ii}\cdot u_{ij}$\\
 $l_{ij}=a_{ij}-\displaystyle \sum_{p=1}^{j-1}l_{ip}\cdot u_{pj}$ &$u_{ij}=\frac{1}{l_{ii}}(a_{ij}- \displaystyle \sum_{p=1}^{i-1}l_{ip}\cdot u_{pj})$ \\
  \hline
\end{tabular}



\end{frame}
%\begin{frame}{Faktoryzacja Crouta}
%Kolejność obliczeń w algorytmie Crouta:

%$$
%\begin{bmatrix}
%1 & 5 & 9 & 13\\
%2 & 6 & 10 & 14\\
%3 & 7 & 11 & 15\\
%4 & 8 & 12 & 16
%\end{bmatrix}
%\begin{array}{lllll}
%| & - & - &- & 2 \\
%| & | & - & - & 4 \\
%| & | & | & - & 6 \\
%| & | & | \\
%1 & 3 & 5 
%\end{array}
%$$
%\begin{flushright}

%{\it Zadanie}: Zbadać złożoność, stworzyć algorytm
%%Związek z eliminacją Gaussa:
%$$
%a_{ij}^{(k+1)}\ =\ \underbrace{a_{ij}^{(k)}} -a_{ik}^{(k)}\ \cdot \underbrace{\frac{a_{kj^{(k)}}}{a_{kk}^{(k)}}}
%$$
%$ \hspace{43mm} =\ \ \  l_{ik}\ \ \ \ \ \ \ \ \ \ \ \ \ u_{kj}
%$
%\newline \hspace*{30mm} różnica $\rightarrow$ kumulacja w 1 kroku

%\end{frame}
\begin{frame}{Faktoryzacja Crouta}
\begin{exampleblock}{Ogólnie faktoryzację można zapisać w  postaci $A=LDU$}
Gdzie:
D - diagonalna \hspace{40mm } 
\newline
 L, U - z ''$1$'' na głównej diagonali\\ 
 \vspace{0.5cm}
 Wtedy faktoryzacja:\\
 {\it Doolittle} to  $L(DU)$\\
 {\it Crout} to   $LD(U)$
\newline 
\end{exampleblock}
\end{frame}
\begin{frame}{Faktoryzacja Choleskiego}
Dygresja:
\begin{itemize}
    \item na wektor $x$ można patrzeć jak na jednokolumnową macierz
    \item na wektor transponowany $x^T$ można patrzeć jak na jednowierszową macierz
    \item jeśli stosujemy zasadę mnożenia macierzy (wiersz x kolumna) to $x^T A x$ będzie liczbą (skalarem) - sprawdzić. 
\end{itemize}
Warunki dla faktoryzacji Choleskiego:
\begin{itemize}
    \item Macierz rzeczywista
\item Macierz symetryczna $a_{ij}=a_{ji}$  czyli $A=A^{T}$
\item Dodatnio określona czyli: $x^T A x > 0$ dla każdego wektora $x$
\end{itemize}
\end{frame}
\begin{frame}
  Dla faktoryzacji $A=LDU$:\\

ponieważ macierz $A$ jest symetryczna i dodatnio określona, to mozna pokazać, że:
$$
l_{ij}=u_{ij},\ i\geq j
$$
czyli $U=L^T$ czyli $A=LDL^{T}$

Dodatkowo, jeżeli $A$ jest dodatnio określona, to $d_{ii}>0, \ \mathrm{i}=1,\ldots,\mathrm{n}$

Czyli można utworzyć $D^{\frac{1}{2}}$, takie że
$D^{\frac{1}{2}}D^{\frac{1}{2}}=D$, gdzie $d^{\frac{1}{2}}_{ii}=\sqrt{d_{ii}}$

\begin{exampleblock}{}
$$
A=(LD^{\frac{1}{2}})(D^{\frac{1}{2}}L^{T})=\overline{L}\ \overline{L}^{T}
$$
\end{exampleblock}
i ta faktoryzacja nosi nazwę Choleskiego

\end{frame}
\begin{frame}{Faktoryzacja Choleskiego}
Dla faktoryzacji:
$$A=\overline{L}\ \overline{L}^{T}$$
  Wyrazy na przekątnej $\overline{L}$ obliczamy:\\
  $$l_{kk}=\sqrt{(a_{kk}-\sum_{s=1}^{k-1}{l_{ks}^2})}$$ \\
  Pozostałe, dla $i>k$:
  $$l_{ik}=\frac{(a_{ik}-\sum_{s=1}^{k-1}{l_{is}l_{ks}})}{l_{kk}}$$ \\
\end{frame}

%\begin{frame}{Computational sequences}
%\begin{exampleblock}{Factorization method:}
%Ability to compute each entry in $L,U$ at one time, accumulating the sum repeated storage and retrieval of 
%\end{exampleblock}
 

%\end{frame}
%\begin{frame}{Computational sequences}
%$+$ - entries used (dane użyte)

%$*$ - entries changed (dane zmienione)

%%\textbf{ a) Gaussian elimination}
%$$
%U_{11}\ U_{12}\ U_{13}\ U_{14}\ U_{15}
%$$
%$$
%L_{21}\ U_{22}\ U_{23}\ U_{24}\ U_{25}
%$$
%$$
%L_{31}\ L_{32}\ |U_{33}^{+}\ |U_{34}^{+}\ |U_{35}^{+}
%$$
%$$
%L_{41}\ L_{42}\ a_{43}^{(3)*}\ a_{44}^{(3)*}\ a_{45}^{(3)*}
%$$
%$$
%L_{51}\ L_{52}\ a_{53}^{(3)*}\ a_{54}^{(3)*}\ a_{55}^{(3)*}
%$$

%\end{frame}
%\begin{frame}{Computational sequences}
%$+$ - entries used (dane użyte)

%$*$ - entries changed (dane zmienione)


%\textbf{a) row Doolittle LU decomposition}
%$$
%|U_{11}^{+}\ |U_{12}^{+}\ |U_{13}^{+}\ |U_{14}^{+}\ %|U_{15}^{+}
%$$
%$$
%|L_{21}^{+}\ |U_{22}^{+}\ |U_{23}^{+}\ |U_{24}^{+}\ %|U_{25}^{+} 
%$$
%$$
%a_{31}^{*}\ a_{32}^{*}\ a_{33}^{*}\ a_{34}^{*}\ a_{35}^{*}
%$$
%$$
%a_{41}\ a_{42}\ a_{43}\ a_{44}\ a_{45} 
%$$
%$$
%%$$

%\end{frame}
%\begin{frame}{Computational sequences}
%$+$ - entries used (dane użyte)

%$*$ - entries changed (dane zmienione)


%\textbf{a) Crout LU decomposition}
%$$
%L_{11}\ U_{12}\ |U_{13}^{+}\ |U_{14}^{+}\ |U_{15}^{+} 
%$$
%$$
%L_{21}\ L_{22}\ |U_{23}^{+}\ |U_{24}^{+}\ |U_{25}^{+} 
%$$
%$$
%|L_{31}^{+}\ |L_{32}^{+}\ a_{33}^{*}\ a_{34}^{*}\ %a_{35}^{*}
%$$
%$$
%|L_{41}^{+}\ |L_{42}^{+}\ a_{43}^{*}\ a_{44}\ a_{45}
%$$
%$$
%|L_{51}^{+}\ |L_{52}^{+}\ a_{53}^{*}\ a_{54}\ a_{55}
%$$

%\end{frame}
%\begin{frame}{Computational sequences}

%\textbf{Computational sequences for LDLT decomposition}

%$+$ - entries used (dane użyte)

%$*$ - entries changed (dane zmienione)

%\textbf{a) symetric Gaussian elimination}
%$$
%\begin{array}{lllll}

%d_{11} \\
%l_{21} & d_{22}\\
%l_{31} & l_{32} & d_{33}\\
%l_{41} & l_{42} & a_{43} & a_{44}\\
%l_{51} & l_{52} & a_{53} & a_{54} & a_{55}

%\end{array}
%$$

%\end{frame}
%\begin{frame}{Computational sequences}

%$+$ - entries used (dane użyte)

%$*$ - entries changed (dane zmienione)

%\textbf{b) row Doolittle LDLT decomposition}
%$$
%\begin{array}{lllll}
%d_{11} \\
%l_{21} & d_{22} \\
%a_{31} & a_{32} & a_{33}\\
%a_{41} & a_{42} & a_{43} & a_{44} \\
%a_{51} & a_{52} & a_{53} & a_{54} & a_{55}
%\end{array}
%$$

%\end{frame} 
%\begin{frame}{Computational sequences}
%$+$ - entries used (dane użyte)

%$*$ - entries changed (dane zmienione)

%\textbf{c) Crout LDLT decomposition}
%$$
%%\begin{array}{lllll}
%d_{11}\\
%l_{21} & d_{22}\\
%l_{31} & l_{32} & a_{33}\\
%l_{41} & l_{42} & a_{43} & a_{44} \\
%l_{51} & l_{52} & a_{53} & a_{54} &  a_{55}
%\end{array}
%$$
%\end{frame}
\begin{frame}{Faktoryzacja blokowa}
Cel: jeśli mamy zoptymalizowane operacje na macierzach (BLAS level 3) to należy ich użyć.\\

\textbf{Przypadek 1) L,U - dowolne: }
$$
A=\left(\begin{array}{ll}
A_{11} & A_{12}\\
A_{21} & A_{22}
\end{array}\right)=\left(\begin{array}{ll}
L_{11} & 0\\
L_{21} & L_{22}
\end{array}\right)\cdot\left(\begin{array}{ll}
U_{11} & U_{12}\\
0 & U_{22}
\end{array}\right)
$$
$A_{11}$, . . . - kwadratowe

$L_{11}, L_{22}-$ - trójkątne dolne 

$U_{11}, U_{22}-$ trójkątne górne
\begin{flushright}
$
A_{11}=L_{11}\cdot U_{11}\quad A_{21}=L_{21}\cdot U_{11}
$
\end{flushright}
\begin{flushright}
$
A_{12}=L_{11}\cdot U_{12}\quad A_{22}=L_{21}\cdot U_{12}+L_{22}\cdot U_{22}
$
\end{flushright}

\end{frame}
\begin{frame}{Faktoryzacja blokowa}
Sposób wyznaczania $L$ i $U$:

\begin{itemize}
\item   \textbf{1.} $A_{11}=L_{11}\cdot U_{11}\rightarrow$

$L_{11}, U_{11}$ obliczamy stosując faktoryzację LU dla $A_{11}$
\item \textbf{2.} $ L_{11}\cdot U_{12}=A_{12}\rightarrow$

kolumny $U_{12}$ obliczamy stosując podstawianie wsteczne

\item \textbf{3.} $L_{21}\cdot U_{11}=A_{21}\rightarrow$

wiersze $L_{21}$ obliczamy stosując podstawianie w przód  
\item \textbf{4.}
Tworzymy pomocniczą macierz $\overset{\bullet}{A_{22}}=A_{22}-L_{21}\cdot U_{12}=L_{22}\cdot U_{22}$, a następnie  stosujemy faktoryzację $\overset{\bullet}{A_{22}}$ dla obliczenia $L_{22}$ oraz $U_{22}$
\end{itemize}

\end{frame}
\begin{frame}{Faktoryzacja blokowa}
\textbf{Przypadek 2) L,U- trójkatne, ale z identycznościami na przekątnej w U:}
$$
A=\left(\begin{array}{ll}
A_{11} & A_{12}\\
A_{21} & A_{22}
\end{array}\right)=\left(\begin{array}{ll}
A_{11} & 0\\
A_{21} & \overset{-}{A_{22}}
\end{array}\right)\cdot \left(\begin{array}{ll}
I & \overset{-}{U_{12}}\\
0 & I
\end{array}\right)
$$
$ A_{12}=A_{11}\cdot \overset{-}{U_{12}}\rightarrow$ kolumny  $\overset{-}{U_{12}}$ znajdujemy rozwiązując odpowiednie układy równań

$ \overset{-}{A_{22}}=A_{22}-A_{21}\cdot \overset{-}{U_{12}}\rightarrow$ obliczamy wykonując operacje na macierzach.

\end{frame}
\begin{frame}{Faktoryzacja blokowa}
\textbf{Przypadek 2) L,U- trójkatne, ale z identycznościami na przekątnej w L}
$$
A=\left(\begin{array}{ll}
A_{11} & A_{12}\\
A_{21} & A_{22}
\end{array}\right)=\left(\begin{array}{ll}
I & 0\\
\overset{\bullet \bullet}{L_{21}} & I
\end{array}\right)\cdot \left(\begin{array}{ll}
A_{11} & A_{12}\\
 0 & \overset{\bullet \bullet}{A_{22}}
\end{array}\right)
$$
$ \overset{\bullet \bullet}{L_{21}}\cdot A_{11}=A_{21}\rightarrow$ 
$\overset{\bullet \bullet}{L_{21}}$ znajdujemy rozwiązując odpowiednie układy równań 

$ \overset{\bullet \bullet}{A_{22}}=A_{22}-\overset{\bullet \bullet}{L_{21}}\cdot A_{12}\rightarrow$ znajdujemy stosując operacje macierzowe

zachodzi $\overset{\bullet}{A_{22}}=\overset{-}{A_{22}}=\overset{\bullet \bullet}{A_{22}}$
(Zaniedbując błędy obliczeń)

Macierze te nazywamy uzupełnieniami Schura (Schur complement)
\end{frame}
\begin{frame}{Faktoryzacja blokowa}
Przykładowo pokażemy kolejne kroki obliczeniowe dla przypadku 2 czyli rozkładu
$$
A=\left(\begin{array}{ll}
A_{11} & 0\\
A_{21} & \overset{-}{A_{22}}
\end{array}\right)\cdot \left(\begin{array}{ll}
I & \overset{-}{U_{12}}\\
0 & I
\end{array}\right)
$$



\end{frame}
\begin{frame}{Faktoryzacja blokowa}
Równanie $\mathrm{A}\mathrm{x}=\mathrm{b}$ rozwiązujemy w następujący sposób:\\

\vspace{0.5cm}

$\hspace*{10mm} \overline{L}\cdot\overline{y}=b\  \hspace*{40mm}\overline{U}\cdot x=y$
\scalebox{0.85}{
$\left(\begin{array}{ll}
A_{11} & 0\\
A_{21} & \overline{A_{22}}
\end{array}\right)\left(\begin{array}{l}
y_{1}\\
y_{2}
\end{array}\right) = \left(\begin{array}{l}
b_{1}\\
b_{2}
\end{array}\right)\ \hspace*{10mm} \left(\begin{array}{ll}
I & \overline{U_{12}}\\
0 & I
\end{array}\right)\left(\begin{array}{l}
x_{1}\\
x_{2}
\end{array}\right) = \left(\begin{array}{l}
y_{1}\\
y_{2}
\end{array}\right)$ }

$\hspace*{33.5mm}\downarrow\ \hspace*{51mm} \downarrow$

$\hspace*{7mm}A_{11}\cdot y_{1}\ \hspace*{12.4mm} =\ \hspace*{3mm} b_{1}\ \hspace*{21.5mm} x_{1}\ \hspace*{15mm} = y_{1}-\overline{U_{12}}\cdot x_{2}$

$A_{21}\cdot y_{1}+\overline{A_{22}}\cdot y_{2}\hspace*{2.5mm} =\ \hspace*{3mm} b_{2}\ \hspace*{20.5mm} x_{2}\ \hspace*{23.5mm} y_{2} $

$\hspace*{8mm} \overline{A_{22}}\cdot y_{2} \hspace*{11.5mm} =\ b_{2}-A_{21}\cdot y_{1}\ $




\end{frame}