\section{Interpolacja trygonometryczna}
%%%%%%%%%%%%%%%%
\begin{frame}{Interpolacja trygonometryczna}
\begin{itemize}
    \item Wielomiany algebraiczne nie są dobre do opisu zjawisk okresowych
    \item Rozwiązanie: interpolacja wielomianami opartymi o funkcje  trygonomentyczne
\end{itemize}

		 funkcje okresowe o okresie L spełniają zależność  $g(y + L) = g(y)$\\
	 funkcje  trygonometryczne: okresem jest $2\pi$, po przeskalowaniu:
	$$	x = \frac{2\pi}{L} \cdot y, f(x) = g(\frac{x \cdot L}{2\pi})$$
 okresem $g(y)$ jest $L$, okresem $f(x)$ jest $2\pi$, 
	 $$f(x+2\pi)=g(\frac{(x+2\pi) \cdot L}{2\pi})=g(\frac{x \cdot L}{2\pi}+L)=
	g(\frac{x \cdot L}{2\pi})=f(x)$$
\end{frame}
\begin{frame}{Interpolacja trygonometryczna}
	szukamy wielomianu trygonometrycznego
	\[
		t_{n-1}(x) =\sum\limits_{j = 0}^{n-1} c_j \cdot (e^{ix})^{j}= \sum\limits_{j = 0}^{n-1} c_j \cdot e^{ijx}
	\]
	gdzie (przypomnienie - wzór Eulera)
	$$
	e^{ix}=cos(x)+isin(x)
	$$
	który w \underline{n punktach $x_k \in (0, 2\pi]$} przyjmuje te same wartości, co interpolowana funkcja
	\\ $t_{n - 1}(x_k) = f(x_k), k = 0, 1, \dots, n - 1$
\end{frame}
%%%%%%%%%%%%%%%%
\begin{frame}{Interpolacja trygonometryczna}
Wielomiany trygonometryczne różnią się od algebraicznych tylko wyborem zmiennej:
	\begin{theorem}
		Zadanie interpolacji trygonometrycznej ma jednoznaczne rozwiązanie.
	\end{theorem}
	%\begin{proofs}
	%	\[
		%	\sum\limits_{j = 0}^{n-1} c_j %\cdot \underbrace{(e^{ix_k})^j}_{Z_k} = f(x_k), k = 0, 1, \dots, n-1
		%	\tag{16.13}
	%	\]
	%	$\implies$ macierz układu: macierz Vandermonde'a jest nieosobliwa, \\ jeżeli \underline{$x_k$ są różne}, i:
	%	\[
	%		\det Z = \prod_{k \neq L} (Z_k - Z_L)
	%		\tag{16.14}
	%	\]
	%\end{proofs}
		W praktyce - ważny przypadek szczególny: \\
		n węzłów równoodległych, \underline{$x_k = \frac{2\pi}{n}\cdot k$}, k = 0, 1, \dots, n - 1 \\
		\underline{funkcje $e^{ijx}$}, j = 0, 1, \dots, n - 1 tworzą \underline{układ ortogonalny} w sensie iloczynu skalarnego zdefinowanego:
		$$
			\langle f|g \rangle = \sum\limits_{k = 0}^{n-1} f(x_k) \cdot g^* (x_k), x_k = \frac{2\pi}{n} \cdot k, k = 0, 1, \dots, n-1
		$$
		\end{frame}
		\begin{frame}{Interpolacja trygonometryczna}
		A dokładniej:
		\[
			\langle e^{ijx}|e^{ilx} \rangle = \sum\limits_{k = 0}^{n - 1}e^{ijx_k}  \cdot e^{-ilx_k} = n \cdot \delta_{j,l}=
			\begin{cases}
				0 & j \neq l \\
				n & j = l
			\end{cases}
		\]
		Dla
		$$
		x_k = \frac{2\pi}{n} \cdot k, k = 0, 1, \dots, n-1
		$$
		Gdzie $\delta_{j,l}$ to delta Kroneckera
		$$
		\delta_{j,l}=\left\{\begin{array}
		{l} 0,\ j\neq l\ \ \\
		1,\ j=l\ 
        \end{array}\right. 
        \text{ dla $j,l \in \{0,1,..,n-1\}$} $$

\end{frame}
\begin{frame}{Interpolacja trygonomentryczna}
Dowód:
		\[
			\langle e^{ijx}|e^{ilx} \rangle = \sum\limits_{k = 0}^{n-1} e^{ijx_k}  \cdot e^{-ilx_k} = \sum\limits_{k = 0}^{n-1} e^{i(j - l) \frac{2\pi k}{n}}(\star)
		\]
		\underline{dla $j = l$}
		\[
			(*)= \sum\limits_{k = 0}^{n-1} e^0 = n
		\]
		\underline{dla $j \neq l$}
		\[
		(*)= \sum\limits_{k = 0}^{n-1} \underbrace{e^{\frac{i(j - l)2\pi}{n} \cdot k}}_{\substack{
				\text{-szereg geom.} \\
				\text{-n-wyrazów}
			}} = a_0\frac{1 - q^n}{1 - q} = \frac{1 - \overbrace{e^{\frac{i(j - l)2\pi}{n} \cdot n}}^{cos((j - l)2\pi)+isin((j - l)2\pi)=1}}{1 - e^{\frac{i(j -  l)2\pi}{n}}} = 0
		\]
		$a_0 = 1; q = e^{\frac{i(j -  l)2\pi}{n}}$
\end{frame}

%%%%%%%%%%%%%%%%
\begin{frame}{Interpolacja trygonometryczna}
	Jakie powinny być współczynniki wielomianu interpolacyjnego $c_j$?
	
	\begin{enumerate}[1$^\circ$]
		\item
		\begin{align*}
			\langle t_{n-1}(x)|e^{ilx} \rangle = \Bigg\langle \sum\limits_{j = 0}^{n-1} c_j \cdot e^{ijx} \Bigg|e^{ilx} \Bigg\rangle = \sum\limits_{j = 0}^{n-1} c_j \cdot n \cdot \delta_{jl} = \\ = c_l \cdot n \implies \underline{c_l = \frac{1}{n} \langle t_{n-1}(x)|e^{ilx} \rangle}
		\end{align*}
		\item
		\begin{align*}
			\langle t_{n-1}(x)|e^{ilx} \rangle= \sum\limits_{k = 0}^{n-1}  \underbrace{t_{n-1}(x_k)}_{\substack{\text{=wartości} \\  \text{funkcji}}} \cdot e^{-ilx_k} =  \sum\limits_{k = 0}^{n-1} f(x_k) \cdot e^{-ilx_k} 
		\end{align*}
	\end{enumerate}
\end{frame}
\begin{frame}{Interpolacja trygonometryczna}	
 \\
	czyli:
	\[
		\underline{c_j = \frac{1}{n} \sum\limits_{k = 0}^{n-1} f(x_k) \cdot e^{-ijx_k}, j = 0, 1, \dots, n-1}
	\]
	\begin{flushleft}
		Dwa etapy obliczeń: \\
		\underline{\textit{analiza Fouriera:}} \\
		dla danych liczb zespolonych $f(x_k), k = 0, 1,\dots, n - 1$ szukamy $c_j, j = 0, 1, \dots, n-1$ \\
		\bigskip
		\underline{\textit{synteza Fouriera:}} \\
		mając liczby $c_j, j = 0, 1, \dots, n-1$ szukamy
		\[
			f(x) = \sum\limits_{j = 0}^{n - 1} c_j \cdot e^{ij\frac{2\pi}{n} \cdot k}, k = 0, 1, \dots, n-1
		\]
	\end{flushleft}
	\end{frame}
	\begin{frame}{Interpolacja trygonometryczna}
	\begin{flushleft}
		$\implies$ obie:
		\begin{itemize}
			\item dyskretne
			\item wzajemnie odwrotne
		\end{itemize}
		\textcolor{blue}{Podsumowanie:}$\newline$
		klasyczny algorytm: $(f = A \cdot C)$
		\[ n^2
		\begin{cases}
			\text{-zespolonych mnożeń,} \\
			\text{-zespolonych dodawań,} \\
			\text{-obliczeń } e^{-i\frac{2\pi kj}{n}}
		\end{cases}
		\]
		$\newline$
		Wada $\rightarrow$ duża złożoność operacji
	\end{flushleft}	
\end{frame}