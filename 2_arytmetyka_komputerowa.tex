\PassOptionsToPackage{lowtilde}{url}
\documentclass[aspectratio=43,english]{beamer} %If you want to create Polish presentation, replace 'english' with 'polish' and uncomment 3-th line, i.e., '\usepackage{polski}'
\usepackage[utf8]{inputenc}
\usepackage{polski} %Uncomment for Polish language
\usepackage{babel}
\usepackage{listings} %We want to put listings

\mode<beamer>{ 	%in 'beamer' mode
	\hypersetup{pdfpagemode=FullScreen}		%Enable Full screen mode
	\usetheme{JuanLesPins} 		%Show part title in right footer
	%\usetheme[dark]{AGH}                 		%Use dark background
	%\usetheme[dark,parttitle=leftfooter]{AGH}  	%Use dark background and show part title in left footer
}
\mode<handout>{	%in 'handout' mode
	\hypersetup{pdfpagemode=None}
	\usepackage{pgfpages}
  	\pgfpagesuselayout{4 on 1}[a4paper,border shrink=5mm,landscape]	%show 4 slides on 1 page
  	\usetheme{boxes}
  	\addheadbox{structure}{\quad\insertpart\hfill\insertsection\hfill\insertsubsection\qquad} 	%content of header
 	\addfootbox{structure}{\quad\insertauthor\hfill\insertframenumber\hfill\insertsubtitle\qquad} 	%content of footer
}

\AtBeginPart{ %At begin part: display its name
	\frame{\partpage}
}


%%%%%%%%%%% Configuration of the listings package %%%%%%%%%%%%%%%%%%%%%%%%%%
% Source: https://en.wikibooks.org/wiki/LaTeX/Source_Code_Listings#Using_the_listings_package
%%%%%%%%%%%%%%%%%%%%%%%%%%%%%%%%%%%%%%%%%%%%%%%%%%%%%%%%%%%%%%%%%%%%%%%%%%%%
\lstset{ %
  backgroundcolor=\color{white},   % choose the background color
  basicstyle=\footnotesize,        % the size of the fonts that are used for the code
  breakatwhitespace=false,         % sets if automatic breaks should only happen at whitespace
  breaklines=true,                 % sets automatic line breaking
  captionpos=b,                    % sets the caption-position to bottom
  commentstyle=\color{green},      % comment style
  deletekeywords={...},            % if you want to delete keywords from the given language
  escapeinside={\%*}{*)},          % if you want to add LaTeX within your code
  extendedchars=true,              % lets you use non-ASCII characters; for 8-bits encodings only, does not work with UTF-8
  frame=single,	                   % adds a frame around the code
  keepspaces=true,                 % keeps spaces in text, useful for keeping indentation of code (possibly needs columns=flexible)
  keywordstyle=\color{blue},       % keyword style
  morekeywords={*,...},            % if you want to add more keywords to the set
  numbers=left,                    % where to put the line-numbers; possible values are (none, left, right)
  numbersep=5pt,                   % how far the line-numbers are from the code
  numberstyle=\tiny\color{gray},   % the style that is used for the line-numbers
  rulecolor=\color{black},         % if not set, the frame-color may be changed on line-breaks within not-black text (e.g. comments (green here))
  showspaces=false,                % show spaces everywhere adding particular underscores; it overrides 'showstringspaces'
  showstringspaces=false,          % underline spaces within strings only
  showtabs=false,                  % show tabs within strings adding particular underscores
  stepnumber=2,                    % the step between two line-numbers. If it's 1, each line will be numbered
  stringstyle=\color{cyan},        % string literal style
  tabsize=2,	                   % sets default tabsize to 2 spaces
  title=\lstname,                  % show the filename of files included with \lstinputlisting; also try caption instead of title
                                   % needed if you want to use UTF-8 Polish chars
  literate={?}{{\k{a}}}1
           {?}{{\k{A}}}1
           {?}{{\k{e}}}1
           {?}{{\k{E}}}1
           {�}{{\'o}}1
           {�}{{\'O}}1
           {?}{{\'s}}1
           {?}{{\'S}}1
           {?}{{\l{}}}1
           {?}{{\L{}}}1
           {?}{{\.z}}1
           {?}{{\.Z}}1
           {?}{{\'z}}1
           {?}{{\'Z}}1
           {?}{{\'c}}1
           {?}{{\'C}}1
           {?}{{\'n}}1
           {?}{{\'N}}1
}
%%%%%%%%%%%%%%%%%
\setcounter{tocdepth}{1}

\newcommand\tab[1][0.5cm]{\hspace*{#1}}


\newcommand{\setcontributors}[1]{
	\let\oldmaketitle\maketitle
	\renewcommand{\maketitle}{
		\begin{frame}
			\oldmaketitle

			\noindent
				\begin{minipage}{0.4\textwidth}
						\footnotesize{\textbf{Contributors}}\\
						\scriptsize{#1}
						% \footnotesize{\textbf{Source code}}\\
						% 	\tab \scriptsize{\href{https://github.com/AGH-MOwNiT-2017/lectures}{\texttt{github.com/AGH-MOwNiT-2017/lectures}}}

				\end{minipage}
				\hfill%
				\begin{minipage}{0.45\textwidth}\raggedleft% adapt widths of minipages to your needs
					\includegraphics[width=25px, height=25px]{img/title/dice}
					\includegraphics[width=60px, height=35px]{img/title/ki}
					\includegraphics[width=30px, height=30px]{img/title/agh}

				\end{minipage}%


		\end{frame}
	}
}


\title{Metody Obliczeniowe w Nauce i Technice}
\author{Marian Bubak, Katarzyna Rycerz}
\date{}
\institute[AGH]{
	Department of Computer Science\\
	AGH University of Science and Technology\\
	Krakow, Poland\\
	\href{mailto:kzajac@agh.edu.pl}{\texttt{kzajac@agh.edu.pl}}\\
	% \href{http://www.icsr.agh.edu.pl/~mownit/}{\texttt{icsr.agh.edu.pl/$\sim$mownit}}
	\href{http://dice.cyfronet.pl/}{\texttt{dice.cyfronet.pl}}

}

%%%%%%%%%%%%%%%%
\usepackage{amsmath}
\usepackage{setspace}
\usepackage{scrextend}
%%%%%%%%%%%%%%%%

\subtitle{Wprowadzenie, arytmetyka komputerowa}
\setcontributors{Maciej Trzebiński\\Mikołaj Biel\\Rafał Stachura}

\begin{document}
  	\maketitle
	%%%%%%%%%%%%%%%%
	\begin{frame}{Outline}
		\tableofcontents
	\end{frame}
	%%%%%%%%%%%%%%%%
	\section{Metody numeryczne wprowadzenie}
%%%%%%%%%%%%%%%%
\begin{frame}{Wprowadzenie}

	\begin{itemize}
		\item metody numeryczne to sposoby rozwiązywania złożonych problemów matematycznych za pomocą podstawowych operacji arytmetycznych,
		\item wykorzystywane gdy badany problem:
		\begin{itemize}
		    \item 	nie ma w ogóle rozwiązania analitycznego (danego 
		    wzorami)
		    \item obliczenie na podstawie wzoru otrzymanego analitycznie ma dużą złożoność
		    \item obliczenie na podstawie wzoru otrzymanego analitycznie  jest źle uwarunkowane numerycznie
		\end{itemize}
	\item otrzymywane  wyniki są na przybliżone, 
	\item dokładność obliczeń może być z góry określona i dobiera się ją zależnie od potrzeb. 
	\end{itemize}
     
\end{frame}
%%%%%%%%%%%%%%%%
\begin{frame}{Literatura}

	\begin{itemize}
		     \item teoria: D. Kincaid, W. Cheney, Analiza numeryczna
		     \item praktyka: Piotr Krzyżanowski, Obliczenia inżynierskie i naukowe
		     \item prosty poradnik: W. Kordecki, K. Serwat, Metody numeryczne dla informatyków
		   
	\end{itemize}
     
\end{frame}
	%%%%%%%%%%%%%%%%
	\section{Numeryczna reprezentacja liczb całkowitych}
%%%%%%%%%%%%%%%%
\begin{frame}{Reprezentacja stałopozycyjna (integer)}

	\begin{itemize}
		\item np. kod U2: na d+1 bitach reprezentowane dokładnie liczby
			\begin{center}
				\[
   					 L \in [-2^d, 2^d-1]
    			\]
			\end{center}
			Uzupełnieniem dwójkowym liczby x zapisanej za pomocą n bitów nazywamy liczbę równą $2^n–x_{(U2)}$
		\item gdy argumenty i wynik reprezentowane stałopozycyjnie, to działania na nich: $+$, $-$, $\cdot$, $/ (div)$, $/ (mod)$ są wykonywane dokładnie
	
		
	\end{itemize}
     
\end{frame}
%%%%%%%%%%%%%%%%
\begin{frame}    
    
    \textbf{Zastosowania:}
    \begin{itemize}
    	\item sprawy walutowe, operacje monetarne
    	\item procesory graficzne np. Sony Nintendo
    	\item rozmiary czcionek w calach np. w TeX
    	\item libfixmath - implementacja biblioteki stałoprzecinkowej w C
    \end{itemize}
    $\newline$
%    \textbf{Zalety i wady:}
%   \begin{itemize}
%    	\item (+) szybkość i prostota
%    	\item (-) mała skalowalność
%    \end{itemize}
\end{frame}
%%%%%%%%%%%%%%%%
	%%%%%%%%%%%%%%%%
	\section{Numeryczna reprezentacja liczb rzeczywistych}
%%%%%%%%%%%%%%%%
\begin{frame}{Reprezentacja zmiennoprzecinkowa (float)}
    Należy pamiętać o ułomności reprezentacji zbioru liczb rzeczywistych $\mathbb{R}$ w rzeczywistym świecie skończonych komputerów.
    \begin{block}{}
    $F$ - zbiór liczb zmiennoprzecinkowych (floating-point):\newline
    \begin{columns}
        \column{0.45\linewidth}
            $\beta$ - podstawa,\newline
            $t$ - dokładność,\newline
            $L, U$ - zakres wykładnika\newline
        \column{0.45\linewidth}
            $d_i$ - liczby całkowite,
            $0 \le d_i \le \beta - 1, i=1,...,t$
            $L \le e \le U$
    \end{columns}
    $x \in F$ ma wartość:
    \[
    x = \pm \underbrace{\left(\frac{d_1}{\beta} + \frac{d_2}{\beta^2} + ... + \frac{d_t}{\beta^t}\right)}_\text{mantysa} \cdot \beta^{\overbrace{e}^\text{cecha}}
    = \pm \sum_{i=1}^{t} \frac{d_i}{\beta^i} \cdot \beta^e
    \]        
    System $F$ jest {\it unormowany}, gdy $\forall_{x \ne 0}\ d_1 \ne 0$.
    \end{block}

\end{frame}

%%%%%%%%%%%%%%%%
\begin{frame}{Reprezentacja zmiennoprzecinkowa (float)}
    \begin{block}{Ważne}
    $F$ nie jest kontinuum - więcej - jest skończony o liczbie elementów wyrażonych wzorem:
    \[
    2 \cdot \left(\beta - 1 \right) \cdot \beta^{t-1} \cdot \left( U - L + 1 \right) + 1
    \]
    \begin{flushright}
    \end{flushright}
    \end{block}
    \textcolor{blue}{Wyjaśnienie:}
    \begin{itemize}
    	\item 2 $\rightarrow$ znak liczby $\pm$
    	\item  $(\beta - 1)$ $\rightarrow$ podstawa, na pierwszym bicie nie ma zera
    	\item $\beta^{t-1}$ $\rightarrow$ pozostałe t-1 bitów z przedziału ${0,\dots,\beta-1}$
    	\item $(U - L + 1)$ $\rightarrow$ zakres wykładnika
    	\item 1 $\rightarrow$ zero
    \end{itemize}
    
\end{frame}
%%%%%%%%%%%%%%%%
\begin{frame}{Reprezentacja zmiennoprzecinkowa (float)}
    Elementy $F$ nie są równomiernie rozłożone na osi:
    $\beta = 2, t = 3, L = -1, U = 2$ \hspace{5mm} (33 elementy):
    \begin{center}
    \includegraphics[width=0.8\linewidth]{img/2/2_1_axis}
    \end{center}

    Każdy element $F$ reprezentuje cały przedział liczb $\mathbb{R}$\newline 
    $x$ - l. rzeczywista $\in$ zakresu $F$,\newline
    $fl(x)$ - reprezentacja zmiennoprzecinkowa liczby $x$

    \[
    \left| \frac{fl(x) - x}{x} \right| \le \frac{1}{2} \beta^{1-t}
    \]

    \begin{flushright}
        {\it Zadanie:} sprawdzić
    \end{flushright}
\end{frame}
%%%%%%%%%%%%%%%%
\begin{frame}{Reprezentacja zmiennoprzecinkowa (float)}
    \begin{alertblock}{Uwaga}
        $0.1$ - często krok w algorytmach\newline
        Czy 10 kroków o długości $0.1$ to to samo co 1 krok = $1.0$?\newline\newline
        W systemie o podstawie $\beta = 2$ - {\bf nie!}
        \[
        0.1_{10} = 0.0(0011)_2 = 0.0(12)_4 = 0.0(6314)_8 = 0.199999..._{16}
        \]

        Reprezentacja $0.1$ urywa się po $t$ cyfrach. Dodanie 10 tak uzyskanych liczb nie da dokładnie $1.0$.
    \end{alertblock}
    
    \begin{alertblock}{Porównania w arytmetyce float}
    Zamiast przyrównywać wartości należy sprawdzać, czy otrzymany błąd pomiędzy wartością obliczoną, a oczekiwaną jest mniejszy od zadanego $\epsilon$.
    \end{alertblock}
\end{frame}
%%%%%%%%%%%%%%%%
\subsection{Dokładność reprezentacji zmiennoprzecinkowej}
\begin{frame}{Reprezentacja zmiennoprzecinkowa (float)}
    \[
    x = s \cdot 2^c \cdot m
    \]
    \begin{center}
    s $\leftarrow$ sign,\ c$\leftarrow$ cecha,\ m $\leftarrow$ mantysa
    \end{center}
    \[
    m = \sum_{i=1}^{\infty} e_i \cdot 2^{-i}
    \]\[
    e_i = \left\{ 
              \begin{array}{ll}
                  0 \\
                  1
              \end{array}
        \right.
    \]

    \begin{block}{Reprezentacja mantysy}
        $$
        m_t = \underbrace{\sum_{i=1}^{t}e_i \cdot 2^{-i}}_\text{t-bitowa mantysa} \ + \underbrace{e_{t+1} \cdot 2^{-t}}_\text{zaokrąglenie}
        $$
    \end{block}
\end{frame}
%%%%%%%%%%%%%%%%
\begin{frame}{Reprezentacja zmiennoprzecinkowa (float)}
    a)Błąd reprezentacji zmiennoprzecinkowej - zaokrąglenie w dół\newline
	$$
		fl(x)^{-} = \pm \sum_{i=1}^{t}{\frac{d_{i}}{\beta_{i}}}\ \cdot \  \beta^e
	$$
    \centering
    \begin{tabular}{|*{5}{p{.75cm}|}}
        \hline
        t & t+1 & t+2 & ... &  \\ \hline
          & 0   & 1   & ... & 1 \\ \hline
    \end{tabular}
    \[
    m = \underbrace{\sum_{i=1}^{t} e_i \cdot 2^{-i} + 0 \cdot 2^{-(t+1)}}_{m_t} +
        \underbrace{\sum_{i=t+2}^{\infty} 1 \cdot 2^{-i}}_{
            \frac{1}{2^{t+1}} = 2^{-(t+1)}
        }
    \] \[
    m - m_t = 2^{-(t+1)}
    \]
\end{frame}
%%%%%%%%%%%%%%%%
\begin{frame}{Reprezentacja zmiennoprzecinkowa (float)}
    b)Błąd reprezentacji zmiennoprzecinkowej - zaokrąglenie w górę\newline
	$$
		fl(x)^{+} = \pm \sum_{i=1}^{t}{\frac{d_{i}}{\beta_{i}}}\ \cdot \  \beta^e \ \pm \frac{d_{t+1}}{\beta_{t+1}}
	$$	
    \centering
    \begin{tabular}{|*{5}{p{.75cm}|}}
        \hline
        t & t+1 & t+2 & ... &  \\ \hline
          & 1   & 0   & ... & 0 \\ \hline
    \end{tabular}
    \[
    m = \sum_{i=1}^{t} e_i \cdot 2^{-i} + 2^{-(t+1)}
    \]\[
    m_t = \sum_{i=1}^{t} e_i \cdot 2^{-i} + 2^{-t}
    \] \[
    m_t - m = 2^{-(t+1)}  \ \ \left| \frac{m - m_t}{m} \right| \le \frac{2^{-(t+1)}}{1/2} = 2^{-t}
    \]
\end{frame}
%%%%%%%%%%%%%%%%
\begin{frame}
	\textbf{Standaryzacja:}
	\begin{itemize}
		\item Reprezentacja zmiennoprzecinkowa została ustandaryzowana w celu ujednolicenia obliczeń
		\item IEEE754 - 1985; IEEE754 - 2008 
		\item treść standardu IEEE754-2008:$\newline$ \url{www.dsc.ufcg.edu.br/~cnum/modulos/Modulo2/IEEE754_2008.pdf}
		\item wywiad z jednym z twórców standardu IEEE754-1985 (William Kahan):\ \url{www.goo.gl/AeNLrd}
	\end{itemize}
	\textbf{Istotne pojęcia:}$\newline$
	\textit{ukryta jedynka, normalizacja mantysy, liczby zdenormalizowane}
\end{frame}
%%%%%%%%%%%%%%%%

	%%%%%%%%%%%%%%%%
	\section{Operacje zmiennopozycyjne}
%%%%%%%%%%%%%%%%
\begin{frame}{Operacje zmiennopozycyjne}
    $x, y \in F$ \newline
    $x + y \in^? F$

    \begin{center}
    \includegraphics[width=0.8\linewidth]{img/2/2_1_axis}
    \end{center}
    Problemy: \newline
    
    $\frac{5}{4} + \frac{3}{8} \notin F$ - ze względu na ,,gęstość'' elementów $F$.\newline
    
     $\frac{7}{2} + \frac{7}{2} \notin F$ - {\it overflow (nadmiar)} 
\end{frame}
%%%%%%%%%%%%%%%%
\begin{frame}{Operacje zmiennopozycyjne}
    
    Oznaczamy:
    $\oplus$ - dodawanie zmiennopozycyjne, \newline
    $fl(x)$ - reprezentacja zmiennoprzecinkowa \newline
    
    działania arytmetyczne na  reprezentacjach liczb rzeczywistych muszą być implementowane tak, jakby działanie było wykonywane dokładnie i tylko wynik reprezentowano w zbiorze liczb maszynowych:\newline
    
    $x \oplus y = fl(x + y)$ dla $x + y$ z zakresu $F$\newline

   Problem: jesli dodajemy liczby odległe od siebie, to ta mniejsza może "zniknąć", mimo, że przed dodawaniem była reprezentowalna 
    \[
    a + b = 2^{c_a} \cdot \left( m_a + m_b \cdot 2^{-(c_a-c_b)} \right) \Rightarrow
    \left\{ 
        \begin{array}{ll}
            dla |b| \le 1/2 \cdot 2^{-t} \cdot |a| \\
            fl(a + b) = a
        \end{array}
    \right.
    \]
\end{frame}
%%%%%%%%%%%%%%%%
\begin{frame}{Operacje zmiennopozycyjne}
    $x \cdot y$ rzadko $\in F$, bo:
    \begin{itemize}
    \item $x \cdot y$ ma $2 \cdot t$ lub $2 \cdot t - 1$ cyfr znaczących,
    \item {\it overflow} - bardziej prawdopodobny
    \item {\it underflow} - bardziej prawdopodobny
    \item $\oplus, \odot$
        \begin{itemize}
        \item są {\it przemienne}
        \item nie są {\it łączne}, {\it rozdzielne}
        \end{itemize}
    \end{itemize}
\end{frame}
%%%%%%%%%%%%%%%%
\begin{frame}{Operacje zmiennopozycyjne}
    Ogólnie:
    \[
    fl(a \square b) = \overbrace{rd(a \square b)}^{\text{inne oznaczenie}} = (a \square b) \cdot (1 + \varepsilon)
    \] \[
    \varepsilon = \varepsilon(a, b, \square), \varepsilon \le \beta^{1-t}
    \] \[
    \square = +, -, \cdot, /
    \]

    \begin{block}{Definicja}
        {\bf Maszynowe $\varepsilon$} - najmniejsza liczba zmiennoprzecinkowa, dla której jeszcze: \[
        1 \oplus \varepsilon > 1
        \]
        Wartość maszynowego $\epsilon$ określa precyzję obliczeń numerycznych wykonywanych na liczbach zmiennoprzecinkowych.
    \end{block}

    Zwykle wystarcza znajomość $\varepsilon' = 2^k \cdot \varepsilon, k \approx 1, 2, 3, ...$ 
\end{frame}
%%%%%%%%%%%%%%%%

	%%%%%%%%%%%%%%%%
	\section{Zadanie, algorytm}
\begin{frame}{Zadanie, algorytm}
Aby badać wpływ zaburzeń danych na wynik, przyjmujemy pewne założenia (prawdziwe dla wiekszości zadań numerycznych):
    \begin{block}{Definicja}
    {\bf Zadanie}\newline
         dla danych    $\vec{d} = \left( d_1, d_2, ..., d_n \right) \in R_d$\newline
         znaleźć wynik  $\vec{w} = \left( w_1, w_2, ..., w_m \right) \in R_w$ taki, że
                $\vec{w} = \varphi(\vec{d})$
                
        
        $R_d$, $R_w$ - skończenie wymiarowe, unormowane przestrzenie kartezjańskie\newline
        $\varphi: D_0 \subset R_d \rightarrow R_w$ - odwzorowanie ciagłe
    \end{block}
    \begin{block}{Definicja}
        {\bf Algorytm A w klasie zadań $\{\varphi, D\}$} jest to sposób wyznaczenia wyniku $\vec{w} = \varphi(\vec{d})$ dla $d \in D \subset D_0$, z dokładną realizacją działań, tj. w zwykłej arytmetyce
    \end{block}
\end{frame}
%%%%%%%%%%%%%%%%
\begin{frame}{Realizacja algorytmu w arytmetyce float - $fl(A(\vec{d}))$}
    Zastąpienie:
    \begin{itemize}
        \item $d, x, ..$
        \item arytmetyki
    \end{itemize}
    Przez:
    \begin{itemize}
        \item $rd(d), rd(x), ..$
        \item arytmetykę float
    \end{itemize}
\end{frame}
%%%%%%%%%%%%%%%%
\begin{frame}{Założenia o reprezentacji danych i wyników}

    W realizacji w arytmetyce float oczekujemy, że $\vec{d}$ oraz $\vec{w}$ będą reprezentowane z małymi błędami 
    \begin{block}{Założenia}
    \[
    || \vec{d} - rd(\vec{d}) ||
    \le 
    \varrho_d ||\vec{d} ||
    \] \[
    || \vec{w} - rd(\vec{w}) ||
    \le 
    \varrho_w ||\vec{w} ||
    \] \[
    \varrho_d, \varrho_w = \underbrace{k}_{\text{małe, } \approx 10} \cdot \  \beta^{1-t}
    \]
    \end{block}
    $\newline$

\end{frame}

	%%%%%%%%%%%%%%%%
	\section{Uwarunkowanie zadania ({\it condition of a problem})}
%%%%%%%%%%%%%%%%
\begin{frame}{Uwarunkowanie zadania}

	{\it Przyczyna:} zamiast 
    	$d_i \rightarrow rd(d_i) = d_i \cdot (1 + \varepsilon)$,
        $||\varepsilon|| \le \beta^{1-t}$
        
    \begin{block}{Definicja}
    	{\bf Uwarunkowanie zadania} - czułość na zaburzenie danych,
        
        {\bf Wskaźniki uwarunkowania zadania} - wielkości charakteryzujące wpływ zaburzeń danych zadania na zaburzenie jego rozwiązania.
    \end{block}
    
    \begin{block}{Definicja}
    Zadanie nazywamy źle uwarunkowanym, jeżeli niewielkie względne zmiany danych zadania powodują duże względne zmiany jego rozwiązania.
    \end{block}
\end{frame}

\begin{frame}{Wskaznik uwarunkowania zadania}
\begin{itemize}
    \item oznaczamy $cond(\varphi(x))$
    \item znaczenie: jeśli dane znamy z błędem względnym nie większym niż $\epsilon$ to błąd względny wyniku  obliczenia nie jest większy niż $\epsilon$ $\cdot$  $cond(\varphi(x))$
    \item np. jeśli $cond(\varphi(x))=100$, a dane reprezentowane są z błedem $2^{-23} \approx 10^{-7}$ to błąd względny wyniku jest nie większy niż $10^{-7} \cdot 100 = 10^{-5}$ (pięć cyfr wyniku jest dokładnych)
\end{itemize}
\end{frame}

%%%%%%%%%%%%%%%%
\begin{frame}{Uwarunkowanie zadania}
	\begin{exampleblock}{Przykład}
    \begin{columns}
    	\column{.45\linewidth} 
        	\centering
            $\vec{x} \cdot \vec{y} = \sum_{i=1}^n x_i \cdot y_i \neq 0$
        \column{.45\linewidth} 
                $x_i \rightarrow x_i \cdot (1 + \alpha_i)$ \\
                $y_i \rightarrow y_i \cdot (1 + \beta_i)$
    \end{columns}
    \begin{gather*}
    	\underbrace{\left| \frac{
        	\sum_{i=1}^n x_i \cdot (1 + \alpha_i) \cdot y_i \cdot (1 + \beta_i) - \sum_{i=1}^n x_i \cdot y_i
        }{
        	\sum_{i=1}^n x_i \cdot y_i
        }\right|}_\text{błąd względny}
        \approx \\ \approx
        \left| \frac{
        	\sum_{i=1}^n \cdot x_i \cdot y_i \cdot \left( \alpha_i + \beta_i \right)
        }{
        	\sum_{i=1}^n x_i \cdot y_i
        }\right|
        \le
        \max \left| \alpha_i + \beta_i \right| \cdot \underbrace{\frac{
        	\sum_{i=1}^n \left| x_i \cdot y_i \right|
        }{
        	\left| \sum_{i=1}^n x_i \cdot y_i \right|
        }}_{cond(\vec{x} \cdot \vec{y})}
    \end{gather*}
    gdy wszystkie $x_i, y_i$ tego samego znaku $\Rightarrow cond(\vec{x} \cdot \vec{y}) = 1$
    \end{exampleblock}
\end{frame}
%%%%%%%%%%%%%%%%
\begin{frame}{Uwarunkowanie zadania}

	{\bf Poprawa:}
    \begin{itemize}
    	\item silniejsza arytmetyka,
        \item użycie zadania równoważnego
    \end{itemize}
\end{frame}
%%%%%%%%%%%%%%%%
	%%%%%%%%%%%%%%%%
	\section{O uwarunkowaniu zadania - inaczej}
%%%%%%%%%%%%%%%%
\begin{frame}{Jakościowo}
	Układ dwóch równań graficznie: \\
    niezależnie od jakości ołówka (algorytm) - lewa strona - dokładniej.
    
    \vspace{.5cm}
    \begin{columns}
    \column{.5\linewidth}
    	\centering   \includegraphics[width=.7\linewidth]{img/2/2_2_well_conditioned}
    \column{.5\linewidth}
    	\centering   \includegraphics[width=.7\linewidth]{img/2/2_3_ill_conditioned}
    \end{columns}
    \vspace{.5cm}
    \begin{columns}
    \column{.5\linewidth}
    	\centering   well conditioned
    \column{.5\linewidth}
    	\centering   ill conditioned
    \end{columns}
\end{frame}
%%%%%%%%%%%%%%%%
\begin{frame}{Ilościowo}

	{\it Zadanie:} wyznaczenie $f(x)$,\ przy założeniu: $x^{*}$ - blisko $x$ \\
	$\newline$
    Współczynnik uwarunkowania:
    
    \vspace{.5cm}
    \centering
    \begin{tabular}{r c}
    	ogólnie: & \(
            cond(f(x)) = \lim_{x^{*} \to x} \frac{
                \left| \frac{
                    f(x) - f(x^{*})
                }{
                    f(x)
                } \right|
            }{
                \left| \frac{
                    x - x^{*}
                }{
                    x
                } \right|
            } = \left| \frac{
                x \cdot f'(x)
            }{
                f(x)
            } \right|
        \)\\
        
        $f(x) = \sqrt{x}$: & \(
        	cond(f(x)) = \left| \frac{
                x \cdot \frac{1}{2 \sqrt{x}}
            }{
                \sqrt{x}
            }\right| = \frac{1}{2}
        \) \\
        
        $f(x) = \frac{1}{1 - x}$: & \(
            cond(f(x)) = \frac{ \left|
                x \cdot \frac{1}{
                    (1-x)^2
                } \right| 
            }{ \left| 
                \frac{1}{1 - x}
            \right| } = \frac{x}{1 - x}
          	\) \\
            & \(
            x = (1 + 10^{-6}) \Rightarrow cond(f(x)) \approx 10^6 \hspace{.5cm} (!)
            \)
    \end{tabular}
\end{frame}
%%%%%%%%%%%%%%%%
	%%%%%%%%%%%%%%%%
	\section{Uwarunkowanie zadania - przykład}
%%%%%%%%%%%%%%%%
\begin{frame}{Uwarunkowanie zadania - przykład}
	\begin{exampleblock}{Przykład 1}
    	\[
        	(x-2)^2 = 10^{-6} \Rightarrow x_\text{1, 2} = 2 \mp 10^{-3}
        \]
        {\bf ALE:} zmiana stałej o $10^{-6} \rightarrow$ to zmiana $x_\text{1, 2}$ o $10^{-3}$!
    \end{exampleblock}
\end{frame}
%%%%%%%%%%%%%%%%
\begin{frame}{Uwarunkowanie zadania - przykład}
	\begin{exampleblock}{Przykład 2 - Wilkinson (1963)}
    Szukamy zer wielomianu:
    	\vspace{.1cm}
        \(
        	p(x) = (x-1)(x-2) \cdot ... \cdot (x-19)(x-20) = x^{20} - 210x^{19} + ...
        \) \vspace{.2cm}
        
        przy zaburzeniu tylko jednego współczynnika (przy 
        $x^{19}$ ): 
        $-210 \to -210 +2^{-23}$  $(2^{-23}\approx 1.19 \cdot 10^{-7})$ \\
        szukamy tak naprawdę zer  dla
        \vspace{.1cm}
        \(
        p(x) + 2^{-23} \cdot x^{19}$ 
         \) \vspace{.2cm}
         
        {\bf wynik:} \(
            \left.
            	\begin{array}{ll}
                10 \to 10.095 ... \mp 0.643...i \\
                ... \\
                19 \to 19.502 ... \mp 1.940...i .
                \end{array}
            \right\rbrace \text{10 pierw. zespolonych!}
        \)
        
	\end{exampleblock}
\end{frame}

\begin{frame}{Uwarunkowanie zadania - przykład}
	\begin{exampleblock}{Przykład 2 - Wilkinson (1963)}
    
    
        {\bf powód:} czułość zadania na zaburzenia danych! \\
        
        \centering
        \(
            p(x,\alpha) = x^{20} - \alpha \cdot x^{19} + ... = 0
        \) \\ \vspace{.1cm}
         Jak zmienia się miejsce zerowe $x_i$ w stosunku do zmian współczynnika $\alpha$ ?\\
         
        \(
            \left.
            \frac{ 
                \partial x
            }{
                \partial \alpha
            } 
            \right|_{x=x_i=i}
            \leftarrow \text{miara czułości}
        \) \\ \vspace{.1cm}\\
        
       
           Korzystam z wzoru na pochodną zupełną:\\
        \(
        	\frac{
            	\partial p(x,\alpha)
            }{
            	\partial x
            } \cdot \frac{
            	\partial x
            }{
            	\partial \alpha
            } + \frac{
            	\partial p(x, \alpha)
            }{
            	\partial \alpha
            } = 0
        \)\\
        
        \(
         \frac{
                  \partial x
              }{
                  \partial \alpha
              } = - \frac{
                  \frac{\partial p}{\partial \alpha}
              }{
                 \frac{\partial p}{\partial x}
              }
              \)
	\end{exampleblock}
\end{frame}
%%%%%%%%%%%%%%%%
\begin{frame}{Uwarunkowanie zadania - przykład}
	\begin{exampleblock}{Przykład 2 - Wilkinson (1963)}
    \begin{columns}
    	\begin{column}{.6\linewidth}
          \centering \includegraphics[width=\linewidth]{img/2/2_4_wilkinson_plot}
          \[
              \frac{
                  \partial x
              }{
                  \partial \alpha
              } = - \frac{
                  \frac{\partial p}{\partial \alpha}
              }{
                 \frac{\partial p}{\partial x}
              } =  \frac{
                  x^{19}
              }{
                  \sum_{i=1}^{20} \prod_{j=1, j \neq i}^{20} (x-j)
              }
          \]
          \[ 
              \left.
                  \frac{\partial x}{\partial \alpha}
              \right|_{x=i} = \frac{
                  i^{19}
              }{
                  \prod_{j=1, j \neq i}^{20}(i-j)
              }, i = 1, 2, ..., 20
          \]
    	\end{column}
        \begin{column}{.3\linewidth}
          \begin{tabular}{r | l}
              i & $\left.
                  \frac{\partial x}{\partial \alpha}
              \right|_i$ \\
              \hline
               1 & $-8.2 \cdot 10^{-18}$ \\
               2 & $ 8.2 \cdot 10^{-11}$ \\
               5 & $-6.1 \cdot 10^{-1}$  \\
               6 & $ 5.8 \cdot 10^{1}$   \\
               8 & $ 6.0 \cdot 10^{4}$   \\
              10 & $ 7.6 \cdot 10^{6}$   \\
              15 & $-2.1 \cdot 10^{9}$   \\
              19 & $-3.1 \cdot 10^{8}$   \\
              20 & $ 4.3 \cdot 10^{7}$   \\
          \end{tabular}
        \end{column}
    \end{columns}
    \end{exampleblock}
\end{frame}
%%%%%%%%%%%%%%%%
	%%%%%%%%%%%%%%%%
	$  $\section{Algorytmy numerycznie poprawne}
\begin{frame}{Algorytmy numerycznie poprawne}
	\begin{block}{Definicja}
		{\bf Algorytmy numerycznie poprawne} - takie, które dają rozwiązania będące nieco zaburzonym dokładnym rozwiązaniem zadania o nieco {\it zaburzonych} danych. Są to algorytmy najwyższej jakości.
        
        {\bf Dane nieco zaburzone} - zaburzone na poziomie reprezentacji.
	\end{block}
\end{frame}
%%%%%%%%%%%%%%%%
\begin{frame}{Algorytmy numerycznie poprawne - definicje}
	\begin{block}{Definicja}
    	Algorytm A jest {\bf numerycznie poprawny} w klasie zadań $\left\{ \varphi, D \right\}$, jeżeli istnieją stałe $K_d, K_w$ takie, że:
        \begin{itemize}
        	\item $\forall \vec{d} \in D$,
            \item dla każdej dostatecznie silnej arytmetyki $\left( \beta^{1-t} \right)$
        \end{itemize}
        $\exists \tilde{d} \in D_0$ taki, że:
        
        {\centering
        	$|| \vec{d} - \tilde{d} || \le \varrho_d \cdot K_d \cdot || \vec{d} ||$ \\ \vspace{.1cm}
            $|| \varphi(\tilde{d}) - fl(A(\vec{d})) || \le \varrho_w \cdot K_w \cdot || \varphi(\tilde{d}) ||$ \\}

        $\varphi(\tilde{d})$ - dokładne rozwiązanie zadania o zaburzonych danych $\tilde{d}$ \\
        $K_w, K_d$ - wskaźniki kumulacji algorytmu A w klasie zadań $\left\{\varphi, D\right\}$
        $K_w, K_d$:
        \begin{itemize}
        	\item dla dowolnych danych klasy $\left\{ \varphi, D \right\}$,
            \item minimalne $\rightarrow$ jakość A.
        \end{itemize}
	\end{block}
\end{frame}
%%%%%%%%%%%%%%%%
\begin{frame}{Algorytmy numerycznie poprawne - definicje}
	\begin{block}{Definicja}
		{\bf Użyteczne algorytmy} - gdy wskaźniki kumulacji rzędu liczby działań
	\end{block}
\end{frame}
%%%%%%%%%%%%%%%%
\begin{frame}[fragile]{Algorytmy numerycznie poprawne - przykład}
	\begin{exampleblock}{Przykład 1 - Iloczyn skalarny}
		Numeryczna poprawność algorytmu $\vec{a} \cdot \vec{b} = \sum_{i=1}^{n} a_i \cdot b_i$
        
\begin{lstlisting}[escapechar=|]
  |$A(\vec{a}, \vec{b}):$|
  s := 0;
  for i:=1 to n do s := s + |$a_1 \cdot b_i$|;
\end{lstlisting}
	\end{exampleblock}
\end{frame}
%%%%%%%%%%%%%%%%
\begin{frame}{Algorytmy numerycznie poprawne - przykład}
	\begin{exampleblock}{Przykład 1 - Iloczyn skalarny }
		{\bf Realizacja algorytmu}: $fl(A(\vec{a}, \vec{b}))$:
        \begin{enumerate}
        	\item dane - reprezentacje \\
                \hspace{1cm} $a_i \to \hat{a_i} = rd(a_i) = a_i \cdot (1 + \alpha_i)$ \\
                \hspace{1cm} $b_i \to \hat{b_i} = rd(b_i) = b_i \cdot (1 + \beta_i)$
        	\item działania - przybliżone, $fl$ \\
            	np. $i = 1, 2, 3:$ \\ 
                \begin{addmargin}[1em]{0em}
                $
                    fl(A(\vec{a}, \vec{b})) = \{
                        [
                            \hat{a_1} \cdot \hat{b_1} \cdot (1 + \varepsilon_1) +
                            \hat{a_2} \cdot \hat{b_2} \cdot (1 + \varepsilon_2)  
                        ]
                        \cdot (1 + \delta_2) + \hat{a_3} \cdot \hat{b_3} \cdot (1 + \varepsilon_3)
                    \} \cdot (1 + \delta_3); 
                $ \\
                $\delta_1 = 0$ \\
                
                \end{addmargin}
            	i ogólnie:
                \begin{addmargin}[1em]{0em}
                $
                	fl(A(\vec{a}, \vec{b})) =
                	\sum_{i=1}^{n} a_i \cdot (1+\alpha_i) \cdot 
                    b_i \cdot (1 + \beta_i) \cdot (1 + \varepsilon_i)
                    \cdot \prod_{j=i}^{n} (1 + \delta_j)
                $                
                \end{addmargin}
        \end{enumerate}
	\end{exampleblock}
\end{frame}
%%%%%%%%%%%%%%%%
\begin{frame}[fragile]{Algorytmy numerycznie poprawne - przykład}
	\begin{exampleblock}{Analiza poprawności numerycznej dla przykładu 1}
	  Istnieją takie $\tilde{a}$ i $\tilde{b}$, dla których da się wskazać $K_d$ i $K_w$.
      Na przykład:
        \begin{itemize}
            \item dla $\tilde{a}$ takiego że $\tilde{a}_i= a_i\cdot(1+\alpha_i)$
             \item dla $\tilde{b}$ takiego że $\tilde{b}_i= b_i\cdot(1+\beta_i) \cdot (1 + \varepsilon_i)
                    \cdot \prod_{j=i}^{n} (1 + \delta_j)$
        	\item mamy dokładny wynik: $K_w = 0$
            \item ale dla zaburzonych danych $\tilde{a}$ i $\tilde{b}$ spełniających warunki :
            	\begin{addmargin}[1cm]{0cm}
            		$|| \vec{a} - \tilde{a} || \le \beta^{1-t} \cdot || \vec{a} \ ||$ \hspace{.5cm} $(K_{d_1} = 1)$\\
            		$|| \vec{b} - \tilde{b} || \le (n + 1) \cdot \beta^{1-t} \cdot || \vec{b} ||$ \hspace{.5cm} $(K_{d_2} = n+1)$
            	\end{addmargin}
        \end{itemize}
        Tutaj skutki błędów zaokrągleń  interpretujemy jako skutki takiego zaburzenia danych, że otrzymany wynik jest dla tych zaburzonych danych dokładny.\\
        Małe zaburzenia oznaczają, że algorytm jest poprawny numerycznie.
	\end{exampleblock}
\end{frame}


%%%%%%%%%%%%%%%%
	%%%%%%%%%%%%%%%%
	\section{Stabilność numeryczna}
%%%%%%%%%%%%%%%%
\begin{frame}{Stabilnosc Numeryczna}
    \begin{itemize}
        \item algorytmy numerycznie poprawne to algorytmy najwyższej jakości
        \item udowodnienie numerycznej poprawności algorytmu jest często trudne, wymaga znalezienie wskazników kumulacji niezależnych od danych
        \item można spróbować zbadać stabilność (słabszy warunek)
         \item  stabilność to  minimalny wymóg dla algorytmu
        \item badamy, jak duży byłby błąd wyniku, gdyby dane oraz wynik zostały zaburzone na poziomie reprezentacji, ale same   obliczenia byłyby wykonywane dokładnie
    \end{itemize}
\end{frame}
%%%%%%%%%%%%%%%%%%%%%%
\subsection{Przykłady}
%%%%%%%%%%%%%%%%
\begin{frame}{Stabilność numeryczna - przykłady}
	\begin{exampleblock}{Przykład 1}
    \setstretch{.5}
    \fontsize{9}{9}
      \begin{columns}[T]
      	\column{.25\linewidth}
        \vspace{.5cm}
            \[
                e^x = 1 + x + \frac{x^2}{2!} + \frac{x^3}{3!} + ...
            \]\[
                \beta = 10, t = 5, x = -5.5
            \]\newline
            \[
            	e^{x}=\sum_{n=0}^{\infty}\frac{x^{n}}{n!}
            \]
		\column{.65\linewidth}
          \begin{columns}
              \column{.55\linewidth}
                  \begin{align*}
                      e^{-5.5}  	=& &   1&.0 \\
                                   & &  -5&.5 \\
                                   & & +15&.125 \\
                                   & & -27&.730 \\
                                   & & +38&.129 \\
                                   & & -41&.942 \\
                                   & & +38&.446 \\
                                   & & -30&.208 \\
                                   & & +20&.768 \\
                                   & & -12&.692 \\
                                   & &  +6&.9803 \\
                                   & &  -3&.4902 \\
                                   & &  +1&.5997 \\
                                   & & .&..\\
                                   \hline \\[-8pt]
                                   & &   0&.0026363
                  \\[-7pt]
                  \text{\bf ALE  } e^{-5.5} =& & 0&.00408677 \hspace{3pt} !
                  \end{align*}
              \column{.38\linewidth}
                  25 składników \\i brak zmian \\w sumie
          \end{columns}
      \end{columns}
	\end{exampleblock}
\end{frame}
%%%%%%%%%%%%%%%%
\begin{frame}{Stabilność numeryczna - przykłady}

	\begin{exampleblock}{Przykład 1}
        {\bf Przyczyna:} {\it catastrophic cancellation}\\
        \begin{itemize}
            \item odejmowanie bliskich liczb - wynikiem jest mała liczba, mająca dużo zer tam, gdzie jej  "poprzednicy" mieli cyfry znaczące,
            \item taka liczba jest normalizowana, zera zapisywane są w wykładniku (cecha), a cała zawartość mantysy przesuwana jest "w lewo"
            \item nie wiadomo czym zapełnić pojawiające się miejsca w mantysie po prawej stronie (zera lub przypadkowe wartości)
        \end{itemize}
        Po zmianie algorytmu: ($\beta, t$ - j.w.)
        \\[-6pt]
        \[
            e^{-5.5} = \frac{1}{e^{5.5}} = \frac{1}{1 + 5.5 + 15.125 + ...} = \underbrace{0.0040865}_\text{0.007\%!}
        \]
        \\[-8pt]
       % $e^x$ oblicza się:
       % \\[-18pt]
      %  \begin{align*}
      %                   x &= m + f, \text{m - całkowite, } 0 \le f < 1 \\
      %                   e^x &= e^{m+f} = e^m \cdot e^f \\
      %      \text{lub }  e^x &= e^{(1 + \frac{f}{m}) \cdot m} = \left[ e^{1 + \frac{f}{m}} %\right]^m
      %  \end{align*}
    \end{exampleblock}
\end{frame}
%%%%%%%%%%%%%%%%
\begin{frame}{Stabilność numeryczna - przykłady}
	\begin{exampleblock}{Przykład 2}
    \setstretch{.7}
		\[
        	E_n = \int_0^1 x^n \cdot e^{x-1} dx, \hspace{.5cm} n = 1, 2, ...
        \]
        całkowanie przez części:
        \\[-15pt]
		\begin{align*}
			& \int_0^1 x^n \cdot e^{x-1} dx = 
            \left. x^n \cdot e^{x-1}
            \right|_0^1 - \int_0^1 n \cdot x^{n-1} \cdot e^{x-1} dx \\
            & \Rightarrow
            \left\{
            	\begin{array}{ll}
            		E_n = 1-n \cdot E_{n-1}, \hspace{.5cm} n = 2, 3, ... \\
                    E_1 = \frac{1}{e}
            	\end{array}
            \right.
		\end{align*}
        \\[-5pt]
        $\beta = 10, t = 6$
        \\[-13pt]
        \begin{align*}
        	E_1 &\approx 0.367879, \varepsilon \approx 4.412 \cdot 10^{-7} \\
            E_2 &\approx 0.264242 \\
            ... \\
            E_8 &\approx 0.118720 \\
            E_9 &\approx -0.0684800 \hspace{.3cm} !! \hspace{.3cm} \text{Bo: } x^9 \cdot e^{x-1} \ge 0, x \in [0, 1]
        \end{align*}
	\end{exampleblock}
\end{frame}
%%%%%%%%%%%%%%%%
\begin{frame}{Stabilność numeryczna - przykłady}
	\begin{exampleblock}{Przykład 2}
		Powód $\rightarrow \varepsilon(E_1):$
        \begin{align*}
        	\text{w } E_2 &\rightarrow \varepsilon \cdot 2 \\
        	\text{w } E_3 &\rightarrow \varepsilon \cdot 2 \cdot 3 \\
            ... \\
            \text{w } E_9 &\rightarrow \varepsilon \cdot 2 \cdot 3 \cdot 4 \cdot ... \cdot 9 = \varepsilon \cdot 9! = \varepsilon \cdot 362880 \approx 0.1601
        \end{align*}
        i nawet: $-0.06848 + 0.1601 = 0.0916$ - poprawny!
	\end{exampleblock}
\end{frame}
%%%%%%%%%%%%%%%%
\begin{frame}{Stabilność numeryczna - przykłady}
	\begin{exampleblock}{Przykład 2}
    Jak wybrać dobry (stabilny) algorytm?
    \\[-10pt]
    \[
    	E_{n-1} = \frac{1 - E_n}{n}, \hspace{.3cm} n = ..., 3, 2.
    \]
    \\[-8pt]
    Na każdym etapie zmniejszamy błąd $n$ razy - {\bf algorytm stabilny}.
    \\[-20pt]
    \begin{align*}
    	E_n = \int_0^1 x^n \cdot e^{x-1} dx \le \int_0^1 x^n dx =
        \left.
        	\frac{x^{n+1}}{n+1}
        \right|_0^1 = \frac{1}{n+1} \\
        \lim_{n \to \infty} E_n = 0 \\
        E_{20} \approx 0.0 \hspace{.3cm} \text{- błąd } \hspace{.1cm} \frac{1}{20} \\
        E_{19} \rightarrow \frac{1}{20} \cdot \frac{1}{21} \approx 0.0024 \\
        \\[-22pt]
        ... \\
        E_{15} \approx 0.0590176 \rightarrow \text{wartość dokładna na 6 miejscach znaczących.}
    \end{align*}
    \end{exampleblock}
\end{frame}
%%%%%%%%%%%%%%%%
\begin{frame}{Stabilność numeryczna - przykłady}
	\begin{block}{Przykład 2 - definicja}
		Metoda numerycznie jest {\bf stabilna}, jeżeli mały błąd na dowolnym etapie przenosi się dalej z {\it malejącą amplitudą}.
        \[
        	\varepsilon^{n+1} = g \cdot \varepsilon^n \hspace{.2cm} 
            \Rightarrow \text{stabilna:}
            \left| \varepsilon^{n+1} \right| \le \left| \varepsilon^n \right|, \hspace{.2cm} g \text{ - wsp. wzmocnienia}
        \]
	\end{block}
	\textbf{Uwaga:} $\newline$
	Jakość wyniku poprawia się wraz z każdym kolejnym etapem obliczeń.
\end{frame}
%%%%%%%%%%%%%%%%
\subsection{Definicja}
%%%%%%%%%%%%%%%%
\begin{frame}{Optymalny poziom błędu rozwiązania}
    \begin{itemize}
        \item Dokładne rozwiązanie dla dokładnych danych (idealne) 	$ \vec{w} = \varphi(\vec{d}) $
        \item Dane zaburzone na poziomie reprezentacji  $\hat{d}: ||\vec{d} - \hat{d}|| \le \varrho_d ||\vec{d} || $ 
        \item Dokładne rozwiązanie dla zaburzonych danych $\hat{w} = \varphi(\hat{d})$
      \item Zaburzone (na poziomie reprezentacji) rozwiązanie dla zaburzonych danych	$\tilde{w}: ||\hat{w} - \tilde{w} || \le \varrho_w ||\hat{w}||$
    \end{itemize}
   Szacujemy różnicę pomiędzy  idealnym $\varphi(\vec{d})$, a zaburzonym $\tilde{w}$     
  \begin{center}
      $|| \vec{w} - \tilde{w} || \le ||\vec{w} - \hat{w}|| + ||\hat{w} - \tilde{w}|| \le
        	||\vec{w} - \hat{w}|| +\varrho_w ||\hat{w}||
        	\le
        	||\vec{w} - \hat{w}|| +\varrho_w (||\hat{w}-\vec{w}||+||\vec{w}||)
        	\le
        	(1 + \varrho_w) \cdot \max_{\hat{d}} ||\varphi(\vec{d}) - \varphi(\hat{d})|| + \varrho_w ||\vec{w}|| $
  \end{center}     
\end{frame}
\begin{frame}{Optymalny poziom błędu rozwiązania}
Definiujemy optymalny poziom błędu rozwiązania jako:
\begin{center}
    $ P(\vec{d}, \varphi) = \varrho_w ||\vec{w}|| + \max_{\hat{d}} ||\varphi(d) - \varphi(\hat{d})||$
\end{center}

Poziom ten wynika wyłącznie  z przeniesienia błędu  reprezentacji  danych  na  wynik  obliczeń
\end{frame}
%%%%%%%%%%%%%%%%
\begin{frame}{Stabilność numeryczna - definicja}
	\begin{block}{Definicja}
		Algorytm A nazywamy {\bf numerycznie stabilnym} w klasie $\{\varphi, D\}$, jeżeli istnieje stała $K$ taka, że 
        \begin{itemize}
        	\item $\forall \vec{d} \in D$,
            \item dla każdej dostatecznie silnej arytmetyki
        \end{itemize}
        zachodzi:
        \[
        	||\varphi(\vec{d}) - fl(A(\vec{d}))|| \le \underbrace{K}_\text{małe} \cdot P(\vec{d}, \varphi)
        \]
       Algorytm numerycznie stabilny gwarantuje uzyskanie  rozwiązania  z błędem co najwyżej
K razy większym, niż optymalny poziom błędu rozwiązania tego zadania.
	\end{block}
\end{frame}
	%%%%%%%%%%%%%%%%
	%\section{Złożoność obliczeniowa {\it computational complexity}}
%%%%%%%%%%%%%%%%
\begin{frame}{Złożoność obliczeniowa}
	Oprócz jakości algorytmu - ważny jego {\it koszt} - liczba działań arytmetycznych (logicznych) potrzebnych do rozwiązania zadania - algorytmy minimalizujące liczbę działań.
\end{frame}
%%%%%%%%%%%%%%%%
\begin{frame}{Rezultaty}
	\begin{enumerate}
		\item Oszacowanie złożoności obliczeniowej ,,z dołu'': 
		\begin{block}{Twierdzenie}
		Jeżeli zadanie ma $n$ danych istotnych, to minimalna liczba działań arytmetycznych potrzebnych do obliczenia wyniku:
        \[
        	z(\varphi, D) \ge \frac{n}{2}
        \]
        \end{block}
        \item Dla wielu zadań można udowodnić, że minimalna liczba działań w algorytmie numerycznie poprawnym musi być istotnie większa od liczby danych.
        \item Metody optymalne co do $z(\varphi, D)$ znane są dla niewielu, prostych zadań.
	\end{enumerate}
\end{frame}
%%%%%%%%%%%%%%%%
\begin{frame}{Złożoność obliczeniowa - przykład}
	\begin{exampleblock}{Przykład}
        \begin{itemize}
            \item prosty
            \item ilustruje konieczność myślenia {\it do końca} przy wyborze algorytmu
        \end{itemize}
        {\bf Zadanie:}
        \[
        	S = \sum_{i=1}^n (-1)^i \cdot i;
        \]
	\end{exampleblock}
\end{frame}
%%%%%%%%%%%%%%%%
\begin{frame}[fragile]{Złożoność obliczeniowa - przykład}
	\begin{exampleblock}{Przykład}
    	{\bf A1:}
    	\begin{lstlisting}[escapechar=|]
s = 0
do 1 i=1, n
    s = s+(-1)**i*i
continue \end{lstlisting}
		\vspace{-15pt}
    	{\bf A2:}
    	\begin{lstlisting}[escapechar=|]
s = 0
do 1 i=1,n,2
    s = s-1
continue
do 1 i=1,n,2
	s = s+1
continue \end{lstlisting}
		\vspace{-15pt}
	\end{exampleblock}
\end{frame}
%%%%%%%%%%%%%%%%
\begin{frame}[fragile]{Złożoność obliczeniowa - przykład}
	\begin{exampleblock}{Przykład}
    	{\bf A3:}
    	\begin{lstlisting}[escapechar=|]
s = n/2
if (mod(n,2).eq.1) s=-s \end{lstlisting}
		\vspace{-15pt}
        \begin{itemize}
        	\item ilość operacji nie zależy od $n$
            \item nie ma akumulacji błędów
            \item nie grozi overflow
        \end{itemize}
	\end{exampleblock}
\end{frame}
	%%%%%%%%%%%%%%%%
	%\input{2_arytmetyka_komputerowa/2_11_zadania}
	%%%%%%%%%%%%%%%%

\end{document}
