\section{Uwarunkowanie zadania ({\it condition of a problem})}
%%%%%%%%%%%%%%%%
\begin{frame}{Uwarunkowanie zadania}

	{\it Przyczyna:} zamiast 
    	$d_i \rightarrow rd(d_i) = d_i \cdot (1 + \varepsilon)$,
        $||\varepsilon|| \le \beta^{1-t}$
        
    \begin{block}{Definicja}
    	{\bf Uwarunkowanie zadania} - czułość na zaburzenie danych,
        
        {\bf Wskaźniki uwarunkowania zadania} - wielkości charakteryzujące wpływ zaburzeń danych zadania na zaburzenie jego rozwiązania.
    \end{block}
    
    \begin{block}{Definicja}
    Zadanie nazywamy źle uwarunkowanym, jeżeli niewielkie względne zmiany danych zadania powodują duże względne zmiany jego rozwiązania.
    \end{block}
\end{frame}

\begin{frame}{Wskaznik uwarunkowania zadania}
\begin{itemize}
    \item oznaczamy $cond(\varphi(x))$
    \item znaczenie: jeśli dane znamy z błędem względnym nie większym niż $\epsilon$ to błąd względny wyniku  obliczenia nie jest większy niż $\epsilon$ $\cdot$  $cond(\varphi(x))$
    \item np. jeśli $cond(\varphi(x))=100$, a dane reprezentowane są z błedem $2^{-23} \approx 10^{-7}$ to błąd względny wyniku jest nie większy niż $10^{-7} \cdot 100 = 10^{-5}$ (pięć cyfr wyniku jest dokładnych)
\end{itemize}
\end{frame}

%%%%%%%%%%%%%%%%
\begin{frame}{Uwarunkowanie zadania}
	\begin{exampleblock}{Przykład}
    \begin{columns}
    	\column{.45\linewidth} 
        	\centering
            $\vec{x} \cdot \vec{y} = \sum_{i=1}^n x_i \cdot y_i \neq 0$
        \column{.45\linewidth} 
                $x_i \rightarrow x_i \cdot (1 + \alpha_i)$ \\
                $y_i \rightarrow y_i \cdot (1 + \beta_i)$
    \end{columns}
    \begin{gather*}
    	\underbrace{\left| \frac{
        	\sum_{i=1}^n x_i \cdot (1 + \alpha_i) \cdot y_i \cdot (1 + \beta_i) - \sum_{i=1}^n x_i \cdot y_i
        }{
        	\sum_{i=1}^n x_i \cdot y_i
        }\right|}_\text{błąd względny}
        \approx \\ \approx
        \left| \frac{
        	\sum_{i=1}^n \cdot x_i \cdot y_i \cdot \left( \alpha_i + \beta_i \right)
        }{
        	\sum_{i=1}^n x_i \cdot y_i
        }\right|
        \le
        \max \left| \alpha_i + \beta_i \right| \cdot \underbrace{\frac{
        	\sum_{i=1}^n \left| x_i \cdot y_i \right|
        }{
        	\left| \sum_{i=1}^n x_i \cdot y_i \right|
        }}_{cond(\vec{x} \cdot \vec{y})}
    \end{gather*}
    gdy wszystkie $x_i, y_i$ tego samego znaku $\Rightarrow cond(\vec{x} \cdot \vec{y}) = 1$
    \end{exampleblock}
\end{frame}
%%%%%%%%%%%%%%%%
\begin{frame}{Uwarunkowanie zadania}

	{\bf Poprawa:}
    \begin{itemize}
    	\item silniejsza arytmetyka,
        \item użycie zadania równoważnego
    \end{itemize}
\end{frame}
%%%%%%%%%%%%%%%%