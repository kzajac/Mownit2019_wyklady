\section{Metody numeryczne wprowadzenie}
%%%%%%%%%%%%%%%%
\begin{frame}{Wprowadzenie}

	\begin{itemize}
		\item metody numeryczne to sposoby rozwiązywania złożonych problemów matematycznych za pomocą podstawowych operacji arytmetycznych,
		\item wykorzystywane gdy badany problem:
		\begin{itemize}
		    \item 	nie ma w ogóle rozwiązania analitycznego (danego 
		    wzorami)
		    \item obliczenie na podstawie wzoru otrzymanego analitycznie ma dużą złożoność
		    \item obliczenie na podstawie wzoru otrzymanego analitycznie  jest źle uwarunkowane numerycznie
		\end{itemize}
	\item otrzymywane  wyniki są na przybliżone, 
	\item dokładność obliczeń może być z góry określona i dobiera się ją zależnie od potrzeb. 
	\end{itemize}
     
\end{frame}
%%%%%%%%%%%%%%%%
\begin{frame}{Literatura}

	\begin{itemize}
		     \item teoria: D. Kincaid, W. Cheney, Analiza numeryczna
		     \item praktyka: Piotr Krzyżanowski, Obliczenia inżynierskie i naukowe
		     \item prosty poradnik: W. Kordecki, K. Serwat, Metody numeryczne dla informatyków
		   
	\end{itemize}
     
\end{frame}