\section{Stabilność numeryczna}
%%%%%%%%%%%%%%%%
\begin{frame}{Stabilnosc Numeryczna}
    \begin{itemize}
        \item algorytmy numerycznie poprawne to algorytmy najwyższej jakości
        \item udowodnienie numerycznej poprawności algorytmu jest często trudne, wymaga znalezienie wskazników kumulacji niezależnych od danych
        \item można spróbować zbadać stabilność (słabszy warunek)
         \item  stabilność to  minimalny wymóg dla algorytmu
        \item badamy, jak duży byłby błąd wyniku, gdyby dane oraz wynik zostały zaburzone na poziomie reprezentacji, ale same   obliczenia byłyby wykonywane dokładnie
    \end{itemize}
\end{frame}
%%%%%%%%%%%%%%%%%%%%%%
\subsection{Przykłady}
%%%%%%%%%%%%%%%%
\begin{frame}{Stabilność numeryczna - przykłady}
	\begin{exampleblock}{Przykład 1}
    \setstretch{.5}
    \fontsize{9}{9}
      \begin{columns}[T]
      	\column{.25\linewidth}
        \vspace{.5cm}
            \[
                e^x = 1 + x + \frac{x^2}{2!} + \frac{x^3}{3!} + ...
            \]\[
                \beta = 10, t = 5, x = -5.5
            \]\newline
            \[
            	e^{x}=\sum_{n=0}^{\infty}\frac{x^{n}}{n!}
            \]
		\column{.65\linewidth}
          \begin{columns}
              \column{.55\linewidth}
                  \begin{align*}
                      e^{-5.5}  	=& &   1&.0 \\
                                   & &  -5&.5 \\
                                   & & +15&.125 \\
                                   & & -27&.730 \\
                                   & & +38&.129 \\
                                   & & -41&.942 \\
                                   & & +38&.446 \\
                                   & & -30&.208 \\
                                   & & +20&.768 \\
                                   & & -12&.692 \\
                                   & &  +6&.9803 \\
                                   & &  -3&.4902 \\
                                   & &  +1&.5997 \\
                                   & & .&..\\
                                   \hline \\[-8pt]
                                   & &   0&.0026363
                  \\[-7pt]
                  \text{\bf ALE  } e^{-5.5} =& & 0&.00408677 \hspace{3pt} !
                  \end{align*}
              \column{.38\linewidth}
                  25 składników \\i brak zmian \\w sumie
          \end{columns}
      \end{columns}
	\end{exampleblock}
\end{frame}
%%%%%%%%%%%%%%%%
\begin{frame}{Stabilność numeryczna - przykłady}

	\begin{exampleblock}{Przykład 1}
        {\bf Przyczyna:} {\it catastrophic cancellation}\\
        \begin{itemize}
            \item odejmowanie bliskich liczb - wynikiem jest mała liczba, mająca dużo zer tam, gdzie jej  "poprzednicy" mieli cyfry znaczące,
            \item taka liczba jest normalizowana, zera zapisywane są w wykładniku (cecha), a cała zawartość mantysy przesuwana jest "w lewo"
            \item nie wiadomo czym zapełnić pojawiające się miejsca w mantysie po prawej stronie (zera lub przypadkowe wartości)
        \end{itemize}
        Po zmianie algorytmu: ($\beta, t$ - j.w.)
        \\[-6pt]
        \[
            e^{-5.5} = \frac{1}{e^{5.5}} = \frac{1}{1 + 5.5 + 15.125 + ...} = \underbrace{0.0040865}_\text{0.007\%!}
        \]
        \\[-8pt]
       % $e^x$ oblicza się:
       % \\[-18pt]
      %  \begin{align*}
      %                   x &= m + f, \text{m - całkowite, } 0 \le f < 1 \\
      %                   e^x &= e^{m+f} = e^m \cdot e^f \\
      %      \text{lub }  e^x &= e^{(1 + \frac{f}{m}) \cdot m} = \left[ e^{1 + \frac{f}{m}} %\right]^m
      %  \end{align*}
    \end{exampleblock}
\end{frame}
%%%%%%%%%%%%%%%%
\begin{frame}{Stabilność numeryczna - przykłady}
	\begin{exampleblock}{Przykład 2}
    \setstretch{.7}
		\[
        	E_n = \int_0^1 x^n \cdot e^{x-1} dx, \hspace{.5cm} n = 1, 2, ...
        \]
        całkowanie przez części:
        \\[-15pt]
		\begin{align*}
			& \int_0^1 x^n \cdot e^{x-1} dx = 
            \left. x^n \cdot e^{x-1}
            \right|_0^1 - \int_0^1 n \cdot x^{n-1} \cdot e^{x-1} dx \\
            & \Rightarrow
            \left\{
            	\begin{array}{ll}
            		E_n = 1-n \cdot E_{n-1}, \hspace{.5cm} n = 2, 3, ... \\
                    E_1 = \frac{1}{e}
            	\end{array}
            \right.
		\end{align*}
        \\[-5pt]
        $\beta = 10, t = 6$
        \\[-13pt]
        \begin{align*}
        	E_1 &\approx 0.367879, \varepsilon \approx 4.412 \cdot 10^{-7} \\
            E_2 &\approx 0.264242 \\
            ... \\
            E_8 &\approx 0.118720 \\
            E_9 &\approx -0.0684800 \hspace{.3cm} !! \hspace{.3cm} \text{Bo: } x^9 \cdot e^{x-1} \ge 0, x \in [0, 1]
        \end{align*}
	\end{exampleblock}
\end{frame}
%%%%%%%%%%%%%%%%
\begin{frame}{Stabilność numeryczna - przykłady}
	\begin{exampleblock}{Przykład 2}
		Powód $\rightarrow \varepsilon(E_1):$
        \begin{align*}
        	\text{w } E_2 &\rightarrow \varepsilon \cdot 2 \\
        	\text{w } E_3 &\rightarrow \varepsilon \cdot 2 \cdot 3 \\
            ... \\
            \text{w } E_9 &\rightarrow \varepsilon \cdot 2 \cdot 3 \cdot 4 \cdot ... \cdot 9 = \varepsilon \cdot 9! = \varepsilon \cdot 362880 \approx 0.1601
        \end{align*}
        i nawet: $-0.06848 + 0.1601 = 0.0916$ - poprawny!
	\end{exampleblock}
\end{frame}
%%%%%%%%%%%%%%%%
\begin{frame}{Stabilność numeryczna - przykłady}
	\begin{exampleblock}{Przykład 2}
    Jak wybrać dobry (stabilny) algorytm?
    \\[-10pt]
    \[
    	E_{n-1} = \frac{1 - E_n}{n}, \hspace{.3cm} n = ..., 3, 2.
    \]
    \\[-8pt]
    Na każdym etapie zmniejszamy błąd $n$ razy - {\bf algorytm stabilny}.
    \\[-20pt]
    \begin{align*}
    	E_n = \int_0^1 x^n \cdot e^{x-1} dx \le \int_0^1 x^n dx =
        \left.
        	\frac{x^{n+1}}{n+1}
        \right|_0^1 = \frac{1}{n+1} \\
        \lim_{n \to \infty} E_n = 0 \\
        E_{20} \approx 0.0 \hspace{.3cm} \text{- błąd } \hspace{.1cm} \frac{1}{21} \\
        E_{19} \rightarrow \frac{1}{20} \cdot \frac{1}{21} \approx 0.0024 \\
        \\[-22pt]
        ... \\
        E_{15} \approx 0.0590176 \rightarrow \text{wartość dokładna na 6 miejscach znaczących.}
    \end{align*}
    \end{exampleblock}
\end{frame}
%%%%%%%%%%%%%%%%
\begin{frame}{Stabilność numeryczna - przykłady}
	\begin{block}{Przykład 2 - definicja}
		Metoda numerycznie jest {\bf stabilna}, jeżeli mały błąd na dowolnym etapie przenosi się dalej z {\it malejącą amplitudą}.
        \[
        	\varepsilon^{n+1} = g \cdot \varepsilon^n \hspace{.2cm} 
            \Rightarrow \text{stabilna:}
            \left| \varepsilon^{n+1} \right| \le \left| \varepsilon^n \right|, \hspace{.2cm} g \text{ - wsp. wzmocnienia}
        \]
	\end{block}
	\textbf{Uwaga:} $\newline$
	Jakość wyniku poprawia się wraz z każdym kolejnym etapem obliczeń.
\end{frame}
%%%%%%%%%%%%%%%%
\subsection{Definicja}
%%%%%%%%%%%%%%%%
\begin{frame}{Optymalny poziom błędu rozwiązania}
    \begin{itemize}
        \item Dokładne rozwiązanie dla dokładnych danych (idealne) 	$ \vec{w} = \varphi(\vec{d}) $
        \item Dane zaburzone na poziomie reprezentacji  $\hat{d}: ||\vec{d} - \hat{d}|| \le \varrho_d ||\vec{d} || $ 
        \item Dokładne rozwiązanie dla zaburzonych danych $\hat{w} = \varphi(\hat{d})$
      \item Zaburzone (na poziomie reprezentacji) rozwiązanie dla zaburzonych danych	$\tilde{w}: ||\hat{w} - \tilde{w} || \le \varrho_w ||\hat{w}||$
    \end{itemize}
   Szacujemy różnicę pomiędzy  idealnym $\varphi(\vec{d})$, a zaburzonym $\tilde{w}$     
  \begin{center}
      $|| \vec{w} - \tilde{w} || \le ||\vec{w} - \hat{w}|| + ||\hat{w} - \tilde{w}|| \le
        	||\vec{w} - \hat{w}|| +\varrho_w ||\hat{w}||
        	\le
        	||\vec{w} - \hat{w}|| +\varrho_w (||\hat{w}-\vec{w}||+||\vec{w}||)
        	\le
        	(1 + \varrho_w) \cdot \max_{\hat{d}} ||\varphi(\vec{d}) - \varphi(\hat{d})|| + \varrho_w ||\vec{w}|| $
  \end{center}     
\end{frame}
\begin{frame}{Optymalny poziom błędu rozwiązania}
Definiujemy optymalny poziom błędu rozwiązania jako:
\begin{center}
    $ P(\vec{d}, \varphi) = \varrho_w ||\vec{w}|| + \max_{\hat{d}} ||\varphi(\vec{d}) - \varphi(\hat{d})||$
\end{center}

Poziom ten wynika wyłącznie  z przeniesienia błędu  reprezentacji  danych  na  wynik  obliczeń
\end{frame}
%%%%%%%%%%%%%%%%
\begin{frame}{Stabilność numeryczna - definicja}
	\begin{block}{Definicja}
		Algorytm A nazywamy {\bf numerycznie stabilnym} w klasie $\{\varphi, D\}$, jeżeli istnieje stała $K$ taka, że 
        \begin{itemize}
        	\item $\forall \vec{d} \in D$,
            \item dla każdej dostatecznie silnej arytmetyki
        \end{itemize}
        zachodzi:
        \[
        	||\varphi(\vec{d}) - fl(A(\vec{d}))|| \le \underbrace{K}_\text{małe} \cdot P(\vec{d}, \varphi)
        \]
       Algorytm numerycznie stabilny gwarantuje uzyskanie  rozwiązania  z błędem co najwyżej
K razy większym, niż optymalny poziom błędu rozwiązania tego zadania.
	\end{block}
\end{frame}