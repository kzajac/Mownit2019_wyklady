\section{Optymalizacja kombinatoryczna}

%%%%%%%%%%%%%%%%
%	\begin{frame}{Optymalizacja kombinatoryczna }
	%	\begin{exampleblock}{Przykład: Problem komiwojażera}
	%		\begin{itemize}
		%		\item \textbf{In}: %symetryczna macierz odległości $(N*N)$

		%		\item \textbf{Out}: permutacja zbioru $\{1,2,...,N\}$ taka, że np:
		%			$$
		%				L_{min} = \sum_{i=1}^N \{ \underbrace{(|x_i - x_{i+1}| + |y_i - y_{i+1}|)}_\text{odległość w metryce Manhattan} + \underbrace{\lambda(\mu_i - \mu_{i-1})}_\text{funkcja kary (penalty)}\}
		%			$$
		%	\end{itemize}
			%DEAD LINK
			%Demo optymalizacji problemu %komiwojażera (autor: Maciej %Borowiec): %\url{komiwojazer/komiwojazer.html}
	%	\end{exampleblock}

	%\end{frame}
%%%%%%%%%%%%%%%%

	\begin{frame}{Optymalizacja kombinatoryczna}
		%\begin{exampleblock}{Zagadnienia NP-zupełne (nondeterministic polynomial)}
			\begin{itemize}
				\item rozwiązania o złożoności $\sim e^N$
				\item wiele stopni swobody
				\item dyskretne (wykluczone poszukiwanie w kierunku)
				\item funkcja celu  łączy przeciwstawne cele cząstkowe
			\end{itemize}
	%	\end{exampleblock}
	%	Dobry przegląd w \cite{garey}
	%	\begin{thebibliography}{9}
			%\setbeamertemplate{bibliography item}[article]
		%	\bibitem{garey}{M.R. Garey, D.S. Johnson \newblock Computers and Intractability: A Guide to the Theory of NP \newblock Completeness, Freeman, San Francisco, 1979}
	%	\end{thebibliography}
	\end{frame}

%%%%%%%%%%%%%%%%
	\begin{frame}{Przykład: problem Max-Cut}
 \begin{figure}
\includegraphics[scale=0.3]{img/18/barbell.png}
%\caption{Graph to cut}
\end{figure} 
\begin{figure}
\includegraphics[scale=0.18]{img/18/partition_barbell.png}
%\caption{Cutting possibilities}
\end{figure} 
Funkcja kosztu:
    \begin{itemize}
        \item[] $g(x_1, x_2)=x_1+x_2-2x_1 x_2$, $x_1,x_2 \in\{0,1\}$
        \item[] $g(00)=0$ $g(11)=0$
        \item[] $g(01)=1$ $g(10)=1$
    \end{itemize}
    \\
    \small{source: https://grove-docs.readthedocs.io/en/latest/qaoa.html}
\end{frame}
\begin{frame}{Przykład - problemy BILP i QUBO}
\begin{block}{Funkcja kosztu i warunki (zmienne binarne)}
$f(x)=c^Tx$ and  $Ax=b$ $x_i \in \{0,1\}$
\end{block}
\begin{center}
    \arrowdown $C$-diagonal, diag($C$)=$c$
\end{center}
\begin{block}{QUBO - quadratic and uncontraint cost function $f(x)=x^TQx$}
$f(x)=x^TCx+P\underbrace{(Ax-b)^T(Ax-b)}_{\substack{\text{instead of solving } Ax=b \\ \text{we minimize inner product of }
Ax-b
}}$ 
\end{block}
\end{frame}

\subsection{Typowa funkcja celu}
	\begin{frame}{Typowa funkcja celu}
		\begin{itemize}
			\item skalar: wszystkie cele sprowadza do jednego
			\item wiele lokalnych minimów - rzędu $e^N$
			\item w praktyce - potrzebne dobre rozwiązanie - nie musi być to minimum globalne
		\end{itemize}

	\end{frame}

%%%%%%%%%%%%%%%%

	\begin{frame}{Typowa funkcja celu}
		\begin{figure}
			\includegraphics[height=0.9\textheight]{img/18/target_fun}
		\end{figure}
	\end{frame}

%%%%%%%%%%%%%%%%

	

%%%%%%%%%%%%%%%%

	\begin{frame}{Przykład procedury generacji nowych konfiguracji}
		\begin{exampleblock}{TSP \cite{lin}}
			Heurystyka: \textit{iterative improvement} - akceptowalne zmiany zmniejszające funkcję celu
			\begin{enumerate}
				\item odwrócenie kolejności obiegu 5-ciu górnych
				\item wstawienie 5-ciu górnych między 2 dolne
			\end{enumerate}
			$\rightarrow$ pewne ugrzęźnięcie w lokalnym minimum
			\begin{figure}
				\includegraphics[height=0.32\textheight]{img/18/tsp}
			\end{figure}
		\end{exampleblock}
		\begin{thebibliography}{9}
			\setbeamertemplate{bibliography item}[book]
			\bibitem{lin}{S. Lin, B.W. Kernighan, Oper. Res. 21 (1973) 498}
		\end{thebibliography}
	\end{frame}
